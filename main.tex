\documentclass{book}
\usepackage{lmodern}
\usepackage{polyglossia}
\setmainlanguage{english}
\title{Technical Typesetting 101}
\author{Petr Sojka, Vít Novotný}
\date{\today}
\begin{document}
  \frontmatter
    \maketitle
    \tableofcontents
  \mainmatter
    \chapter{Foreword}
      With the advent of the digital age, typesetting has become available to
      virtually anyone equipped with a personal computer. Beautiful documents
      can now be crafted using free and consumer-grade software, which obviates
      the need for the involvement of a professional typesetter. The level
      playing field of the Internet coupled with the rising popularity of
      digital-only documents then allows the aspiring author to bypass the
      publisher as well, if they so wish, without jeopardizing their chance of
      recognition.
      
      This text is intended as a handbook for any author who aspires to write,
      design, typeset and distribute high-quality documents of their own making.
      Each chapter describes one discrete step in the process of the creation of
      a document and consists of two parts. The first part of each chapter is
      dedicated to the historical background surrounding the subject as well as
      the general principles, whereas the second part illustrates the principles
      with real-world examples using common tools. Due to the ephemeral nature
      of software, the usefulness of the examples will invariably wane in time,
      but if we have succeeded with our task, the principles should retain their
      value regardless of the reader's choice of tools.

      The order of the chapters corresponds to the realities of document
      preparation and therefore forms a useful mental model, which allows the
      author to break the process down to manageable parts and tackle each of
      them separately. This also allows for the separation of labour, where
      each part of the process can be completed by a different specialist.
      However, a lone author who is preparing a less complex document or who
      feels confident in their ability to multitask may find it more productive
      to interleave several steps of the document preparation process and this
      is completely legitimate.

    \chapter{Markup}
    \chapter{Design}
    \chapter{Typesetting}
    \chapter{Proofreading}
    \chapter{Printing}
    \chapter{Distribution}
\end{document}
