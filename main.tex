\documentclass[b5paper]{book}
\usepackage{lmodern}
\usepackage{polyglossia}
\setmainlanguage{english}
\usepackage{./main}
\title{Electronic Document Preparation: An Author's Cookbook}
\author{Petr Sojka, Vít Novotný}
\date{\today}
\begin{document}
  \frontmatter
    \maketitle
    \tableofcontents
  \mainmatter
    \chapter{Foreword}
      With the advent of the digital age, typesetting has become available to
      virtually anyone equipped with a personal computer. Beautiful documents
      can now be crafted using free and consumer-grade software, which obviates
      the need for the involvement of a professional typesetter. The level
      playing field of the Internet coupled with the rising popularity of
      digital-only documents then allows the aspiring author to bypass the
      publisher as well, if they so wish, without jeopardizing their chance of
      recognition.
      
      This text is intended as a handbook for any author who aspires to write,
      design, typeset and distribute high-quality documents of their own making.
      Each chapter describes one discrete step in the process of the creation of
      a document % and consists of two parts. The first part of each chapter is
    % dedicated to the historical background surrounding the subject as well as
    % the general principles, whereas the second part illustrates the principles
    % with real-world examples using common tools. Due to the ephemeral nature
    % of software, the usefulness of the examples will invariably wane in time,
    % but if we have succeeded with our task, the principles should retain their
    % value regardless of the reader's choice of tools.
    %
    %%%%%%%%%%%%%%%%%%%%%%%%%%%%%%%%%%%%%%%%%%%%%%%%%%%%%%%%%%%%%%%%%%%
    %%% The structure of the chapters differs from the description. %%%
    %%%%%%%%%%%%%%%%%%%%%%%%%%%%%%%%%%%%%%%%%%%%%%%%%%%%%%%%%%%%%%%%%%%

      The order of the chapters corresponds to the realities of document
      preparation and therefore forms a useful mental model, which allows the
      author to break the process down to manageable parts and tackle each of
      them separately. This also allows for the separation of labour, where
      each part of the process can be completed by a different specialist.
      However, a lone author who is preparing a less complex document or who
      feels confident in their ability to multitask may find it more productive
      to interleave several steps of the document preparation process and this
      is a completely legitimate thing to do.

    \chapter{Writing}
      % Text encodings
      % Text editors
      % Regular expressions
      % Versioning systems

    \chapter{Markup}
      A manuscript can consist of a seamless river of words and still make
      perfect sense to the author. To truly capture its meaning in a clear and
      unambiguous manner, however, the manuscript will often need to be
      supplemented with an additional set of annotations. At a more basic level,
      this refers to the compliance with the orthographic rules, such as the
      correct spelling, hyphenation, capitalization, word breaks and
      punctuation, that are specific to the language of the document. It is not
      unreasonable to expect that this basic compliance should be already met by
      the manuscript. At a higher level, this consists of discovering and
      marking up the inner order and logic of the text, so that the resulting
      document can later be typeset in a way that visually reflects the
      structure of the text. To this end, there exists a wealth of \emph{markup
      languages} that enable the enrichment of plain text documents with
      additional information and labels.

      \section{SGML and its descendants}
        According to \cite{hlava11}, the situation engulfing digital typesetting
        was growing increasingly frustrating for publishers in the 1960s. The
        markup languages used by different typesetting systems varied wildly and
        once a publisher had a large collection of documents typeset via a given
        company, switching to another one could be very costly. The companies
        would often take advantage of this situation, causing their prices to
        skyrocket. As a result of that, a demand for a universal markup language
        emerged.

        This demand was met by a side product of a project developed\footnote{
          More information about the project can be found within the personal
          recollections of its co-author, Charles F. Goldfarb, in
          \cite{goldfarb96,goldfarb97}.
        } at the IBM's Cambridge Scientific Center in the early 1970s. The
        project aimed at imbuing a text editor with the ability to query, edit
        and display documents from a repository to allow the usage of computers
        in legal practice. Very early on in the development process, it became
        clear that the main problem were going to be the markup languages in
        which the documents were written. These languages weren't unified and,
        rather than describing the logical structure of the documents, they
        directly shaped their visual form, which made information retrieval
        impossible without the use of heuristics. To resolve the issue, a
        unifying markup language called the General Markup Language
        (\acronym{GML}) was drafted as a solution to the decribed problem. The
        language was later released to the public in \cite{goldfarb81} and
        finally standardized as the Standard General Markup Language
        (\acronym{SGML}) within the \acronym{ISO} 8879 standard.

        % An example of an SGML document

    \chapter{Design}
    \chapter{Typesetting}
    \chapter{Proofreading}
    \chapter{Printing}
    \chapter{Distribution}
\end{document}
