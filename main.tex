\documentclass[
  a5paper,10pt,           % Basic geometry and type size
  dvipsnames              % To be passed to the xcolor package
]{book}
\IfFileExists{shellesc.sty}{%
  \usepackage{shellesc}}{}% Patch LuaTeX 0.87+ to support `\write18`.
\usepackage{etex}         % Reserve room for 32 additional insertion classes
\reserveinserts{32}       %% that will not be taken away by `\newcount` et al.
\usepackage{polyglossia}  % Hyphenation
\setmainlanguage{english}
\setotherlanguages{french, italian, german, czech}
\usepackage{placeins}     % Float barriers
\usepackage{tikz}         % Diagrams
\usetikzlibrary{trees,arrows}
\usepackage{fancyvrb}     % Manually colored verbatim
\let\oldendVerbatim       %% patched so that there is no extra vertical space
    \endVerbatim          %% after the environment
\def\endVerbatim{%
  \oldendVerbatim
  \vspace{-9pt}}
\usepackage{hologo}       % TeX engine logos
\usepackage{float}        % Non-floating figures
\usepackage{url}          % URLs
\usepackage{tabularx}     % Tables
\usepackage{booktabs}
\usepackage{dcolumn}
\usepackage{latexsym}     % Additional LaTeX symbols
\usepackage{pifont}       % Dingbats
\usepackage[              % Bibliography
  backend=biber,
  style=numeric,
  citestyle=numeric-comp,
  sorting=none,
  backref=true
]{biblatex}
\addbibresource{main.bib}
\usepackage[nottoc]{tocbibind}
\usepackage{makeidx}      % Index
\makeindex
\usepackage{varwidth}     % Variable-width minipages
\usepackage{./main}       % Markup and design
% Chapter 1
\defacronym{ASCII}{the American Standard Code for Information Interchange}
\defacronym{ASA}{the American Standard Association}
\defacronym{IBM}{the International Business Machines Corporation}
\defacronym{FORTRAN}{the FORmula TRANslator}
\defacronym{COBOL}{the COmmon Business-Oriented Language}
\defacronym{ISO}{the International Organization for Standardization}
\defacronym{IEC}{the International Electrotechnical Commission}
\defacronym{UCS}{the Universal multiple-octet coded Character Set}
\defacronym{BMP}{the Basic Multilingual Plane}
\defacronym{UTF}{the \acroshort{UCS} Transformation Format}
% Chapter 2
\defacronym{GML}{the General Markup Language}
\defacronym{SGML}{the Standard General Markup Language}
\defacronym{DTD}{Document Type Declaration}
\defacronym{XML}{the eXtensible Markup Language}
\defacronym{Relax NG}{the REgular LAnguage for \acronym{XML} New Generation}
\defacronym[cite=rfc3987]{IRI}{the Internationalized Resource Identifier}
\defacronym{CSS}{the Cascading Style Sheets language}
\defacronym{TEI}{the Text Encoding Initiative}
\defacronym{MathML}{the Mathematical Markup Language}
\defacronym{SVG}{the Scalable Vector Graphics language}
\defacronym[foreign={\foreign[french]{la Conseil Européen pour la Recherche
  Nucléaire}}]{CERN}{the European Organization for Nuclear Research}
\defacronym{HTML}{the HyperText Markup Language}
\defacronym{W3C}{the World Wide Web Consortium}
\defacronym{XHTML}{the eXtensible Hypertext Markup Language}
\defacronym{RDF}{the Resource Description Framework}
\defacronym{DC}{the Dublin Core}
\defacronym{FOAF}{Friend Or A Foe}
\defacronym{OWL}{the Web Ontology Language}
\defacronym{JSON-LD}{the JavaScript Object Notation for Linked Data}
\defacronym{RDFa}{\acronym{RDF} in attributes}
\defacronym{WYSIWYG}{What You See Is What You Get}
\defacronym{GUI}{Graphical User Interface}
\defacronym{GNU}{\acroflat{GNU} is Not Unix}
\defacronym{API}{Application Programming Interface}
\defacronym{ECMA}{the European Computer Manufacturers Association}
\mdefacronym{ATnT}{AT\scamp T}{the American Telephone and Telegraph corporation}
% Bibliography
\defacronym{USA}{the United States of America}
\defacronym{MA}{Massachusetts}
\defacronym{NY}{New York}
\defacronym{JTC}{the Joint Technical Committee}
\defacronym{SC}{a SubCommittee}
\defacronym{WG}{a Working Group}
\defacronym{RFC}{a Request For Comments}
        % Acronyms
\includeonly{chapters/writing,chapters/markup,chapters/design}
\begin{document}
\frontmatter
\title{Electronic Document Preparation\\Pocket Primer}
\author{Vít Novotný}
\maketitle
\tableofcontents
\mainmatter
\fakechapter{Introduction}
With the advent of the digital age, typesetting has become available to
virtually anyone equipped with a personal computer. Beautiful text documents can
now be crafted using free and consumer-grade software, which often obviates the
need for the involvement of a professional designer and typesetter. The level
playing field of the Internet coupled with the rising popularity of digital-only
documents then allows the author to bypass the publisher as well, if they so
wish, without jeopardizing their chance of recognition.

This aim of this book is to provide a general overview of the tools and
techniques tied with writing, designing, typesetting, and distributing text
documents---one of the principal means of knowledge preservation and transfer
known to man. Each chapter describes one discrete step of document preparation
along with practical examples and references to literature for those interested
in further study.

The chapter are filled with examples that illustrate the subject matter. These
should be consulted whenever the concepts described in the text are unclear to
the reader. Although care was taken not to favor any computing environment,
some examples feature utilities for \Unices. These utilities may or may not have
a suitable counterpart in operating systems such as Windows; To try the
corresponding examples out, the reader is advised to install a free Unix-like
environment---such as Cygwin for Windows---on their computer.

\chapter{Writing}
The essence of a document is the idea it represents. In the case of a text
document, this idea is articulated through speech, which is transcribed using
text, optionally accompanied by figures, and then laid out on a sheet of paper
according to a design. Since the text is typically independent on the design,
whose task is to support and elicit the internal structure of the text, it is
writing that is the logical first step in the text document creation.

\begin{figure}
  \input examples/01/trichter
  \caption{Exceptions that prove the rule about the separation of text and
    design can sometimes be encountered in poetry. Above is \person{Christian
    Morgenstern}'s \foreign[german]{\work{Trichter}}, where the text and its
    form are intimately intertwined.}
\end{figure}

The essentials of writing in any given natural language include \termpl*{grammar
rule}\index{writing rules!grammar}, which specify the structure of spoken
language, and \termpl*{orthographic rule}\index{writing rules!ortography}, which
impose additional requirements on written text. The complexity of either set of
rules depends entirely on the language in question. Some writing systems, such
as the Japanese kanji, are not phonographic and the correspondence between
the spoken words and the written symbols needs to be memorized by the writer on
a word-to-word basis. Other languages may use vastly different grammar rules for
speaking and for writing, which means that a spoken sentence needs to be
translated first before writing down. A writer needs to recognize these
specifics.

On top of grammar and orthographic rules stand \termpl{style guide}, which, in
order to improve consistency, codify how common language patterns are encoded.
More comprehensive style guides---such as \work{the Chicago Manual of Style} or
\work{the Oxford Style Manual}\footnote{
  This document was prepared in accordance with \person{William Strunk}'s
  \work{Elements of Style}, an American English style guide for general use. 
}---often go beyond writing and provide guidelines on design and typesetting as
well, making them an indispensable reference on the editorial tradition.

Above all stand the \termpl*{typographic rule}\index{writing rules!typography},
which specify how the resulting document should be typeset so that it doesn't
disturb the eye of the reader. These, as well as the orthographic rules on
hyphenation, can be left out of consideration during writing, as it is the page
that should be formed around the writing and not the other way around.

\section{Text Processing}
Originally the domain of the pen, the quill, the stylus, and the more recent
typewriter machine, manuscripts of today are produced chiefly using the personal
computer and stored in \termpl{text file}. The discipline of creating and
manipulating digital text is called \term{text processing} and will be the focus
of this section.

\subsection{Character Encoding}
Although computing at its most primal has no use for anything but numbers, it
has nevertheless been accompanied by text from the very outset. Even the
earliest computers from 1950s were programmed with both raw machine code and
the text programming languages of \acronym{FORTRAN} and \acronym{COBOL}. The
digital representation of letters, digits and other characters was initially
closely tied to each specific application and processor architecture, but with
the advent of networking in 1960s, mutual intelligibility became a point of
concern.\iffalse\footnote{
  \Acronym{EBCDIC} by \acronym{IBM} was the default encoding on
  \acroshort{IBM}'s System/360 mainframes and was in active use until the
  introduction of \acronym{PC} in 1981. In countries using Chinese ideographs,
  special encodings, such as Big5, \acronym{JIS}, and \acronym{EUC}, are used to
  this day. For brevity, the text focuses on the main stream of international
  encodings.}\fi
\ \quote{We had over sixty different ways to represent characters in
computers. It was a real Tower of Babel,} explains \cite{brandel99}
\person{Bob Berner}, an American computer scientist who worked at \acronym{IBM}
during 1956--1962 and who drafted \acronym{ASCII}---a
\term{character encoding} from \citeyear{asa63} that unified the industry and
enabled computer networking on large scale.

\subsubsection{ASCII}
In \acronym{ASCII}, every character is represented by a number from zero to 127,
which is transformed to a seven-bit integer called a \term{character code}.
These 128 codes are used to encode \termpl{printable character}---spanning the
letters of the English alphabet, digits, punctuation, and other symbols---and
\termpl{control code}, as depicted in Table \ref{tab:ascii}.  Unlike printable
characters, control codes have no fixed visual representation and they were used
to implement application-specific communication protocols and text formatting;
their precise semantics were defined in the much later standard of
\acroshort{ISO}~646 from \citeyear{iso72} \cite{iso72}. Unconstrained by the
bandwidth and the storage limitations of the 1960s and 1970s, today's
communication protocols and text formats gravitate towards markup constructed
from printable characters, which, unlike control codes, are easy to read and
write by humans.

\begin{table}
  \input examples/01/ascii
  \caption{The \acronym{ASCII} encoding, as specified in the \citeyear{asa86}
    revision of the standard \cite{asa86}.}
  \label{tab:ascii}
\end{table}

\begin{table}
  \input examples/01/utf8
  \caption{The \acroshort{UTF}-8 encoding. Each \textvisiblespace\ represents
    one bit of the \acroshort{UCS} code point in binary.}
  \label{tab:utf8}
\end{table}

\begin{table}
  \input examples/01/utf8-example
  \caption{An example of the \acroshort{UTF}-8 encoding}
  \label{tab:utf8-example}
\end{table}

%%% ASCII, Other Standards
%%%   <https://en.wikipedia.org/wiki/ASCII#Other_standards>
%%%   <https://en.wikipedia.org/wiki/ISO/IEC_8859-2#External_links>
%%%
%%% Character histories: notes on some Ascii code positions
%%%   <http://www.cs.tut.fi/~jkorpela/latin1/ascii-hist.html>
%%%
%%% ASCII: American Standard Code for Information Infiltration
%%%   <http://worldpowersystems.com/J/codes/>
%%%
%%% Encyclopedia of Computer Science, 4th edition
%%%   <https://dl.acm.org/ralston.cfm?CFID=698031209&CFTOKEN=46500462>
%%%
%%% Theoretical Foundation of Regular Expressions and Text Editors
%%%   <http://citeseerx.ist.psu.edu/viewdoc/download?rep=rep1&type=pdf&doi=10.1.1.126.9920>
%%%
%%% A History of Scientific Text Processing at CERN
%%%   <http://ref.web.cern.ch/ref/CERN/CNL/2001/001/tp_history/>
%%%
%%% When and how did text enter the world of computing, eventually to be
%%% standardized as ASCII in 1963? <http://qr.ae/RHFEzE>
%%%
%%% IBM's Early Computers: A Technical History
%%%   <http://www.amazon.com/IBMs-Early-Computers-Technical-Computing/dp/0262523930>

The following properties make it easy to manipulate and reason about character
strings encoded in \acronym{ASCII}:
\begin{itemize}
  \item Each character is represented by exactly seven bits. This makes it easy
    to allocate space for character strings of fixed length, to measure the
    number of characters stored in a memory region, and to perform basic
    operations, such as adjacent character retrieval or text truncation.
  \item Characters are alphabetically ordered. Character strings can therefore
    be collated by comparing character code binary values.
  \item Lowercase and uppercase letters, digits and control codes form
    contiguous ranges of character codes, simplifying classification.
  \item There is precisely one way to encode any printable character. The
    conversion between the lower- and uppercase letters is a matter of
    inverting one bit.
\end{itemize}
This comes at the expense of support for non-English writing systems. As a
temporary workaround, a set of \acroshort{ASCII} derivatives that replaced the
less-needed characters of \# \$ @ [ \textbackslash\ ] \textasciicircum\ ` \{ |
\} and \textasciitilde\ for international characters was specified in the
\acroshort{ISO}~646 standard \cite{iso72} from \citeyear{iso72}.

\subsubsection{Eight-bit Encodings}
With the byte size stabilizing at eight bits, new character encodings emerged
that were based on \acronym{ASCII} and used the additional bit to encode
characters of non-English writing systems while retaining complete backwards
compatibility with \acroshort{ASCII}. Beside the numerous vendor-specific
encodings (called \termpl{code page}), a set of 15 eight-bit encodings covering
all major modern writing systems whose characters fit within the space of 128
additional combinations was standardized in the
\acroshort{ISO}/\acroshort{IEC}~8859 series released during 1986--2001.

  % Show a time diagram of Czech encodings
  %   <http://luki.sdf-eu.org/txt/cs-encodings-faq.html>

Compared to \acronym{ASCII}, eight-bit encodings introduced an additional level
of complexity to text processing:
\begin{itemize}
  \item Each character is exactly eight bits wide. The manipulation with strings
    is therefore as straightforward as with \acronym{ASCII}.
  \item Character strings can no longer be collated by character code
    comparison. Each encoding requires a separate mapping from character codes
    to sorting weights.
  \item Classes of characters, such as uppercase and lowercase letters or
    punctuation, no longer form contiguous ranges and their position varies
    among encodings. This impedes character classification.
  \item Idiosyncrasies, such as the ligature of æ and invisible hyphenation
    hints, are included in several encodings, which makes it more difficult to
    determine character string equivalence. Algorithms for case conversion vary
    among encodings.
  \item There exists no standard mechanism to detect which encoding is being
    used. The distinction needs to be done on the application level using either
    heuristics or additional metadata. Consequently, no standard mechanism
    exists to use different character encodings within a single text document.
\end{itemize}
A portion of this complexity is inherent in the task of encoding the characters
of all modern writing systems, but the overhead caused by the character encoding
fragmentation proved to be unnecessary.

\subsubsection{The Universal Character Set and Unicode}
In the early 1990s, the continual increase in the available bandwidth and
storage led to the creation of the standards of Unicode
\cite{unicode91,unicode92} and \acronym{UCS} in an attempt to create a text
encoding that would contain the characters of all the world's living languages
and succeed \acronym{ASCII} as the \foreign[italian]{lingua franca} of text
interchange.

\Acronym{UCS} is an ever-expanding catalogue of characters from writing systems
both modern and ancient, and symbols ranging from diacritical marks,
punctuation, and ideograms to mahjong tiles, alchemical symbols, and the ancient
Greek musical notation. Each of these characters is assigned a number, called
a \term{code point}, ranging from 0 to 2,147,483,647 (\hexa{7FFFFFFF}) with the
numbers of the most common characters in the range from 0 to 65,535
(\hexa{FFFF}) called \acronym{BMP}. The smallest unit of division in
\acronym{UCS} are \termpl*{block}\acroindex[!block]{UCS}, which contain 256
thematically related characters. \Acronym{UCS} encodings map character numbers
to binary character codes.

Three major encodings\footnote{
  Notable are also the seven-bit encodings of \acroshort{UTF}-7
  \acroindex[!UTF-7@\acroshort{UTF}-7]{UTF} and \inx{Punycode}, which bring
  Unicode support to protocols that were designed with the seven-bit
  \acroshort{ASCII} in mind, such as e-mail.}
are specified in the \acronym{UCS} standard and its amendments
\cite{iso93:am1,iso93:am2}:
\begin{description}
  \item[\acroshort{UTF}-32]\acroindex[!UTF-32@{\acroshort{UTF}-32}]{UTF}Directly
    encodes \acronym{UCS} characters by transforming their code points to
    four-byte integers. \acroshort{UTF}-32 is also known as
    \acroshort{UCS}-4\acroindex[!UCS-4@{\acroshort{UCS}-4}]{UCS}.
  \item[\acroshort{UTF}-16]\acroindex[!UTF-16@{\acroshort{UTF}-16}]{UTF}
    Directly encodes characters within \acronym{BMP} by transforming their code
    points to two-byte integers. Code points in the range from 65,536 to
    1,114,111 (\mbox{\hexa{010000}--\hexa{10FFFF}}) are transformed into pairs
    of two-byte integers, called \termpl{surrogate pair}, ranging from
    55,296 to 57,343 (\mbox{\hexa{DC00}--\hexa{DFFF}}). To enable the
    \acroshort{UTF}-16 encoding, the code points in the surrogate pair range
    will never be assigned \cite[sec.\,3.4]{unicode15}. The same applies to code
    points greater than 1,114,111 (\hexa{10FFFF}), which allows
    \acroshort{UTF}-16 to encode any \acroshort{UCS} character.
  \item[\acroshort{UTF}-8]\acroindex[!UTF-8@{\acroshort{UTF}-8}]{UTF}
    Directly transforms code points ranging from 0 to 127 (\hexa{7F}) to
    one-byte integers. Since the first \acroshort{UCS} block of the
    \acroshort{BMP} matches \acronym{ASCII}, any text encoded in eight-bit
    \acroshort{ASCII} is also encoded in \acroshort{UTF}-8. Code points in the
    range from 127 to 1,114,111 (\mbox{\hexa{00007F}--\hexa{10FFFF}}) are
    transformed into two to four one-byte integers ranging from 128 to 253
    (\mbox{\hexa{80}--\hexa{FD}}). The encoding is illustrated in tables
    \ref{tab:utf8} and \ref{tab:utf8-example}.
\end{description}
\acroshort{UTF}-32 is primarily used for the internal representation of
individual \acronym{UCS} characters inside programs, \acroshort{UTF}-16 fulfills
a similar role in applications that only work with \acronym{BMP}, and
\acroshort{UTF}-8 is used for text storage and interchange. Since 2010, the
majority of text content on the Web is encoded in \acroshort{ASCII} and
\acroshort{UTF}-8 \cite{qsuccess15}.

Originally a competing standard, Unicode underwent a merger with \acronym{UCS}
in version 1.1 and since then, the standards have been kept closely
synchronised. Unicode is a superset of \acronym{UCS}, which defines additional
information about \acronym{UCS} characters---such as their general category,
directionality, case, or numeric value \cite[sec.\,3.5 and ch.\,4]{unicode15}%
---, various text processing algorithms, and implementation guidelines.

\begin{figure}
  \ffootnote{One of the design goals of \acroshort{UCS} was to avoid assigning
    code points to different glyphs that carry the same meaning. As a result,
    the visually distinctive Han characters used in the East Asian countries
    of China, Japan, Korea, and Vietnam were merged into a set of 75,960
    ideograms in a process referred to as the \term{Han Unification}. This
    simplifies text processing, but also makes it impossible to encode a text
    in multiple East Asian languages without having to rely on external markup
    to select appropriate regional fonts. As a result, a derivative of
    \acronym{UCS} that doesn't implement the Han Unification was developed for
    use in operating systems based on \acronym{TRON} and is used in the East
    Asia alongside \acronym{UCS} and region-specific encodings.}
  \input examples/01/unihan
  \caption{Several Han characters in the traditional Chinese, Japanese,
    Korean, and Vietnamese variants}
\end{figure}

Regarding text processing, Unicode and \acronym{UCS} represent a compromise
between the simplicity of the seven-bit \acronym{ASCII} and the heterogeneity of
eight-bit encodings:
\begin{itemize}
  \item If simple text manipulation is preferred over space efficiency, each
    character can be made exactly two or four bytes wide using the
    \acroshort{UTF}-16 and \acroshort{UTF}-32 encodings.
  \item Although character strings can not be collated by a simple character
    code comparison, a collation algorithm is defined in the Unicode
    specification \cite{unicode15:collation} and collation tables for major
    locales \cite{unicode15:cldr} are maintained by the Unicode Consortium.
  \item Classes of characters---such as uppercase letters, lowercase letters,
    numbers, and punctuation---do not form contiguous ranges, but their position
    is directly specified by the standard \cite[sec.\,4.5]{unicode15}.
  \item Although idiosyncrasies---such as ligatures, invisible hyphenation
    hints, and combining characters---are present in \acronym{UCS}, explicit
    normalization algorithms for character string equivalence testing are
    specified by the standard \cite[sec.\,2.12]{unicode15}. An algorithm
    for case conversion is also specified \cite[sec.\,3.13]{unicode15}.
    \index{Unicode!normalization}\index{Unicode!case conversion}
  \item The \ucs[FEFF]{Byte Order Mark} character can be inserted at the
    beginning of a text as a signature of Unicode encodings. As the name
    suggests, the order in which the \hexa{FE} and \hexa{FF} bytes arrive also
    indicates the order of bytes (called \term{endianity}) that was used to
    encode integers. In \acroshort{UTF}-32 and \acroshort{UTF}-16, endianity
    can be chosen arbitrarily by the encoding application. In \acroshort{UTF}-8,
    one-byte integers are used and the notion of endianity is therefore
    meaningless.
\end{itemize}

\begin{figure}
  \input examples/01/combining-chars
  \caption{Some \acroshort{UCS} characters can be either input as a single
    entity or composed from several combining characters. Regarding Unicode
    normalization forms, all of the above representations are canonically
    equivalent.}
\end{figure}

\begin{figure}
  \centerline{\code[sh]{iconv -f latin2 -t utf8 -- old.txt > new.txt}}%
  \caption{Text files can be converted between encodings using the
    \cliutil{iconv} command-line tool. The sample code shows the file
    \filename{old.txt} being converted from the
    \acroshort{ISO}/\acroshort{IEC}~8859-2 encoding to \mbox{\acroshort{UTF}-8}.
    The result of the conversion is stored in the file \filename{new.txt}.}
\end{figure}

%%% Setting collation order in shell sort
%%%   <http://superuser.com/a/414408/136765>
%%%
%%% Unicode 101: An Introduction to the Unicode Standard
%%%   <http://www.interproinc.com/blog/unicode-101-introduction-unicode-standard>
%%%
%%% Unicode Implementation levels
%%%   <http://www.cl.cam.ac.uk/~mgk25/unicode.html#levels>
%%%
%%% Unification of the Unicode Standard and ISO 10646
%%%   <http://www.unicode.org/versions/Unicode1.0.0/V2ch01.pdf>
%%%
%%% Unicode 88 <http://unicode.org/history/unicode88.pdf>
%%% Unicode equivalence <https://en.wikipedia.org/wiki/Unicode_equivalence>
%%%
%%% Plane (Unicode):
%%%   <https://en.wikipedia.org/wiki/Plane_(Unicode)#Basic_Multilingual_Plane>

\begin{figure}[p]
  \centerline{\includegraphics[width=0.75\textwidth]%
    {examples/01/google-pinyin.png}}
  \caption{Text input methods are not limited to keyboard layouts. Software that
    allows for the input of non-Latin characters on a keyboard through reversed
    romanization can often be the best option for writing systems with a large
    number of characters. Above is the \inx{Google Pinyin} input method for the
    Android operating system, which makes it possible to input Chinese
    characters using the \inx{pinyin} phonetic system.}
\end{figure}

\begin{figure}[p]
  \input examples/01/composeKey
  \caption{The \key{Compose}\index{Compose@\displaykey{Compose}} key followed by
    a mnemonic sequence of \acroshort{ASCII} characters produces a
    \acroshort{UCS} character. Although originally a physical key, \key{Compose}
    is not available on modern \acroshort{PC} and Apple keyboards and is usually
    mapped to a less-used key, such as the right \key{Ctrl} or \key{Super} key.
    \key{Compose} is natively supported on \Unices\ using the \inx{X Window
    System}. On other operating systems, support can be added by third-party
    software.}
\end{figure}

\begin{figure}
  \input examples/01/altCodes
  \caption{On the Windows operating system, holding the \key{Alt} key and typing
    a sequence of numbers produces a character with the corresponding number
    from either an \acroshort{IBM} code page, if the number has no leading zero,
    or from a Windows code page otherwise. The code pages vary depending on the
    current locale; in English locales, the \acroshort{IBM} code page~437
    and the Windows code page~1252 are used. After a Windows Registry
    modification, it is also possible to directly produce \acroshort{UCS}
    characters by holding the \key{Alt} key and typing the corresponding
    \acroshort{UCS} code point in hexadecimal.}
\end{figure}

\subsection{Text Input}
To insert text into a document, it is necessary to use an input device. In case
of personal computers, this is typically a computer keyboard and a mouse,
although the ongoing research in the areas of \acronym{SR} and \acronym{OCR}
makes it possible to use a microphone or a tablet as well. On hand-held devices,
the use of either a numeric keypad or a touch-screen is more typical.

An operating system will typically provide one or more input methods for each
input device through a component commonly referred to as the \acronym{IME}. The
\acroshort{ASCII} encoding was developed with typewriters and teleprinters in
mind and, as their direct descendant, the standard computer keyboard provides
support for all \acroshort{ASCII} characters. This doesn't apply to the much
larger \acronym{UCS} and it is the task of an \acronym{IME} to provide a
mechanism for the creation and selection of keyboard layouts that will allow the
user to input any \acronym{UCS} character. Some programs may provide input
methods of their own.

\subsection{Text Editors}
A \term{text editor} is an applications, which can be used to create and modify
text files. Entry-level text editors are often distributed with an operating
system and offer little beyond the ability to load, modify and save text files
in a text encoding of choice. Entry-level text editors with a \acronym{GUI}
include the \inx{Windows Notepad}, the iOS \inx{TextEdit} in plain text mode,
and \inx{Leafpad} for \Linux\ and \acronym{BSD}. Entry-level text editors with a
\acronym{CLI} include \cliutil{joe}, \acroshort{GNU}
\cliutil*{nano}\acroindex[!nano]{GNU}, and \cliutil{pico}.

More advanced text editors come with the support for \termpl{regular expression}
and \term{version control}---which will be covered in sections \ref{sec:regexs}
and \ref{sec:vcs}---, as well as user modules that extend the base
functionality.  Advanced \acronym{GUI} text editors include \inx{Sublime Text},
\inx{Atom}, and \inx{PSPad}. Advanced \acronym{CLI} text editors include
\cliutil{emacs}, \cliutil{vi}, and \cliutil{vim}. The presented \acronym{CLI}
text editors are especially notorious for their learning curve, whose steepness
is only matched by the power these editors grant to those who wield them.

\subsection{Interactive Document Preparation Systems}
Interactive \acropl{DPS}\acroindex[!interactive]{DPS} are a breed of text
editors that produces fully-formatted text documents instead of (or along with)
text files. The reader is advices to avoid interactive \acropl{DPS} that use
proprietary, undocumented, and obscure file formats which lock the user into
using the respective \acronym{DPS} to open the files reliably. Well-defined
interactive \acronym{DPS} file formats include \acronym{PDF}, \acronym{OOXML},
and \acronym{ODF}.

The primary difference between text editors and \acropl{DPS} is the fact that
the user is expected to use the \acronym{DPS} to mark up, design, and typeset the
resulting text document, whereas with plain text files a multitude of choices is
available at each step of the document preparation process. The self-sufficient
nature of \acropl{DPS} may be a time-saving feature for simpler documents, but
in the case of more complex documents, the markup and typesetting capabilities
of a \acronym{DPS} may not be up to par with those of a dedicated tool.
Interactive \acropl{DPS} include \inx{Apache OpenOffice}, \inx{TextEdit},
\inx{Microsoft Word}, \inx{Scribus}, \inx{Adobe InDesign}, \inx{Adobe
FrameMaker} and \inx{QuarkXPress}.

\subsection{Regular Expressions}\label{sec:regexs}
The \term{Chomsky hierarchy} is a classification of text production rules
(called \termpl{formal grammar}), which was proposed \cite{chomsky56} in
\citeyear{chomsky56} by the American linguist \person{Noam Chomsky} in his
endeavor to discover a good formal model for the description of natural
languages. The class of \termpl{regular grammar}, which is the least powerful
of the proposed classes, has properties that make it possible to determine the
grammaticality of a text in constant time and space. \Termpl{regular
expression} are a more intuitive reformulation of regular grammars that a writer
can use to find and replace character strings within text.

Since regular expressions are just a formal model, a software implementation
needs to settle on a concrete syntax. One of the earliest standard syntaxes are
\acronym{BRE} and \acronym{ERE} \cite[part~1,~ch.\,9]{iso93:posix2}, which are
supported by most utilities using regular expressions on \Unices. Both syntaxes
are described in Table \ref{tab:regexs}. More extensive syntaxes include the
\acroshort{GNU} extensions of \acronym{BRE} and \acronym{ERE}, the regex syntag
of the \inx{Perl} programming language, and their dialects and derivatives. For
most of these extended syntaxes, the term \term*{regular} is a misnomer, as the
expressions can be used to build grammars that, according to the Chomsky
hierarchy, aren't regular.\footnote{
  A curious reader with a background in formal language theory should direct
  their attention to Perl self-referencing groups and look-aheads, which make it
  a simple matter to create expressions that match non-context-free formal
  languages.
% It is easy to show that the Perl 5 regex of \code{(?=(a(?-1)?b)c)
% a+(b(?-1)?c)} generates the context-sensitive language of $a^nb^nc^n$ for
% arbitrary characters $a,b,c$ with superscripts denoting repetition.
} To disambiguate the term, these expressions are often called
\termpl*[regexes]{regex}\index{regex|see{regular expression}}.

\begin{table}
  \input examples/01/regexs
  \label{tab:regexs}
\end{table}

%%% POSIX.2-1997: Regular Expressions
%%%   <http://pubs.opengroup.org/onlinepubs/007908799/xbd/re.html>

Many regex syntaxes and the software that implements them were designed for the
processing of \acronym{ASCII} text and may behave in surprising ways, when
confronted with \acronym{UCS} characters. The software may assume that each
character is exactly one byte long and fail to recognize any characters that
occupy several bytes as a single character, or it may assume that all
\acronym{UCS} characters fall within \acronym{BMP} and exhibit the same problem
with characters outside \acronym{BMP}. More subtle, but no less precarious, can
be the lack of support for Unicode case conversion and normalization algorithms,
which makes it difficult to perform robust case-insensitive matching and the
matching of characters that can be encoded in several different ways. The lack
of awareness of the invisible characters that can appear in \acronym{UCS}
text---such as the \ucs[200B]{zero width space}, \ucs[200C]{zero width
non-joiner}, \ucs[200D]{zero width joiner}, and \ucs[FEFF]{zero width no-break
space}---, is also problematic and can lead to false negative matches.
Conversely, modern regex syntaxes that at least partially implement the Unicode
standard for Regular Expressions \cite{unicode13}---such as those of Perl 5.2 or
Java 7---are actively aware of \acronym{UCS} and provide features that enable
the matching of characters based on their general category, numeric value,
directionality, and other properties defined by Unicode, as shown in Table
\ref{tab:unicode-regexs}.

\begin{table}[!tb]
  \input examples/01/unicode-regexs
  \caption{An overview of the elements of the Unicode regex syntax as
    implemented by Perl 5.2 and Java 7. The list of Unicode character properties
    is only demonstrative.}
  \label{tab:unicode-regexs}
\end{table}

The most elementary text processing \acroshort{CLI} tool is \cliutil{grep}%
\footnote{
  The authoritative resource on \cliutil{grep}, \cliutil{sed}, and \cliutil{awk}
  is \citework{dougherty97}, which explains each tool as well as the
  \acroshort{BRE} and \acroshort{ERE} syntaxes in full detail.
}, which makes it possible to search text files for fixed strings and regexes in
default of an advanced text editor. Unless configured otherwise, the tool will
present lines that contain one or more matches to the user. A more advanced
text-processing \acroshort{CLI} tool is \cliutil{sed}, which features a simple
programming language that can be used to arbitrarily transform text files.
\Cliutil{awk} is a \acroshort{CLI} tool that also features a text-processing
programming language, albeit a more advanced one than that of \cliutil{sed}.
Originally developed for the Research Unix during 1973--1977, \cliutil{grep},
\cliutil{sed}, and \cliutil{awk} are available in various flavors for most
operating systems.

%%% The true power of regular expressions
%%%   <https://nikic.github.io/2012/06/15/The-true-power-of-regular-expressions.html>
%%%
%%% Regular Expression Matching Can Be Simple And Fast
%%%   <https://swtch.com/~rsc/regexp/regexp1.html>
%%%
%%% JavaScript has a Unicode problem
%%%   <https://mathiasbynens.be/notes/javascript-unicode>

\section{Version Control}\label{sec:vcs}
When writing a text document, it is ofter useful to have a backup of the
previous versions of files, so that undesirable changes can be reverted whenever
necessary. If more than one person contributes to the document, the ability to
track the authorship of these changes also becomes an asset. At its most
rudimentary, \acronym{VCS} records changes along with a short description and
information about their author. These changes can then be browsed, and reverted.
With a single contributor, \acronym{VCS} are a convenient alternative to manual
version archival. With several contributors, \acronym{VCS} becomes an essential
tool.

\Acronym{VCS} can be dichotomized based on its architecture, which is either
\term*{centralized}\acroindex[!centralized]{VCS} or \term*{decentralized}
\acroindex[!decentralized]{VCS}. Centralized \acronym{VCS} stores
all versions in a repository located on a remote server. Users send new versions
to the server and retrieve existing versions using a client software. The client
software is \term*{thin} in the sense that it does not store more than one
version locally and its operation is fully dependent on the availability of the
server. An example of centralized \acronym{VCS} is \acronym{SVN}\footnote{
  The authoritative resource on \acronym{SVN} is \citework{sussman02},
  affectionately known as \work{the Subversion book}.
}.

By comparison, there is no designated server in decentralized \acronym{VCS} and
the users can send and download new versions directly from one another. The
client software is \term*{thick} in the sense that all users have a local
repository with every existing version, which they can browse and manipulate
at any time. The disadvantages of decentralization include the more complex
workflow, greater storage size requirements and the increased opportunity for
the users not to share their local changes frequently enough, leading to an
increased chance of collisions. Examples of decentralized \acronym{VCS} include
\inx{Git}, \inx{Mercurial}, or \inx{Bazaar}.

\begin{figure}
  \input examples/01/svn
\end{figure}

\begin{figure}
%  \ffootnote{Although it is typical to use a central repository, the
%    decentralized architecture of Git makes it possible for clients to exchange
%    the contents of their repositories directly.  Multiple layers of
%    repositories that automatically exchange the latest updates can also be
%    created for backup and other purposes.}
  \input examples/01/git
\end{figure}

%\begin{figure}
%  \input examples/01/git-svn
%\end{figure}

Although \acronym{VCS} can be used to keep track of any kind of files, they are
especially geared towards text files, which they can easily display along with
changes. However, most interactive \acropl{DPS} do not produce text files, which
can make version control challenging. As a solution, some \acropl{DPS} include
an internal version control functionality that can record changes directly into
output files. Other \acropl{DPS} provide an interface for external \acronym{VCS}
to display changes between two versions of output documents produced by the
\acropl{DPS}\footnote{
  An example would be the graphical \acroshort{SVN} client \inx{Tortoise
  \acroshort{SVN}} that is able to display the changes between two versions
  of Microsoft Word documents using the interface provided by Microsoft Office.
}. A class of its own are web services that enable real-time interactive
collaboration---such as \inx{Word Online} or \inx{Google Documents}.

\begin{figure}
  \includegraphics[width=\textwidth]{examples/01/word.png}\nextimage
  \includegraphics[width=\textwidth]{examples/01/openoffice.png}
  \caption{The built-in \acronym{VCS} of \inx{Microsoft Word} (above) and
    \inx{Apache OpenOffice} (below)}
\end{figure}

\begin{figure}
  \includegraphics[width=\textwidth]{examples/01/tortoise-svn.png}
  \caption{\inx{Tortoise \acroshort{SVN}} is a graphical frontend for
    \acronym{SVN} with the ability to display the difference between two versions
    of a \inx{Microsoft Word} document even though it is not a text file.}
\end{figure}

%%% Git porcelains <http://stackoverflow.com/a/6978402>
%%% Git <-> SVN
%%%   <https://git-scm.com/book/en/v1/Git-and-Other-Systems-Git-and-Subversion>
%%%   <http://www.janosgyerik.com/practical-tips-for-using-git-with-large-subversion-repositories/>
%%%   <http://stackoverflow.com/a/772881/657401>

\chapter{Markup}
A manuscript can be a seamless current of words and still make perfect sense to
an author. To truly capture its meaning in a clear and unambiguous manner,
however, the author will often need to supplement the manuscript with a set of
annotations. At a more fundamental level, this refers to the compliance with the
orthographic rules---such as the correct spelling, capitalization, word breaks,
and punctuation---that are specific to the language of the document. It is not
at all unreasonable to expect that this basic compliance should be already met
by the manuscript. At a higher level, this consists of discovering and marking
up the inner order and logic of the text, so that the resulting document can
later be typeset in a way that visually reflects its structure.

It is not unusual for an author to write and mark up their manuscript at the
same time. Nevertheless, each of the two activities represents a distinct
concept. Writing is the process of breaking ideas down into raw sequences of
words. To mark up these words then is to take and reassemble them back into
meaningful units of linguistic thought.

\index{markup!logical}\index{markup!presentation}
Markup can be created using a variety of \termpl*{markup language}. Aside from
\term*{logical markup}, which captures the logical structure of a document,
markup languages may also provide \term*{presentation markup}, which directly
impacts the visual properties of the document but carries no semantic
information. The usage of presentation markup makes it impossible to separate
the markup from the design and to capture the structure of the document. As a
result, the consistency in the design of each logical part of the document needs
to be ensured manually, and future changes of design become error-prone and
tedious. In this regard, logical markup is to design what style guides are to
writing: a means of ensuring internal consistency that should be used whenever
possible.

\section{Meta Markup Languages}
\subsection{The General Markup Language}
The situation engulfing digital typesetting was growing increasingly frustrating
for publishers in the 1960s. The markup languages used by different typesetting
systems varied wildly and once a publisher had a large collection of documents
typeset via a given company, switching to another one could be a costly
venture. This power imbalance artificially increased the price of digital
typesetting, leading to a demand for a universal markup language.

This demand was met by a project developed\sidenote{
  More information about the project can be found within
  \citework{goldfarb96} and \citework*(SGML: The Reason Why and the First
  Published Hint)[]{goldfarb97:whySGML}.
} at the Cambridge Scientific Center of \acronym{IBM} in the early 1970s. The
project aimed at imbuing a text editor with the ability to query, edit, and
display documents from a central repository to allow the usage of computers in
legal practice. Very early on in the development it became apparent that the
main problem were going to be the markup languages in which the documents were
written. These languages varied wildly and many of them comprised largely
presentation markup, which made information retrieval impossible without heavy
use of heuristics. To resolve these issues, a unifying markup language called
\acronym{GML} was drafted. The language was released~\cite{goldfarb81} to the
public in \citeyear{goldfarb81} and finally standardized in \citeyear{iso86} as
\acronym{SGML}\sidenote{
  The authoritative resource on \acronym{SGML} is \citework{goldfarb91}, which
  includes the full text of the standard bearing extensive annotations.
}.~\cite{iso86}

\Acronym{SGML} documents consist of text mixed with \termpl*{tag}%
\acroindex[!tag]{SGML}, which delimit meaningful sections of the document called
\termpl*{element}\acroindex[!element]{SGML}. Elements may carry additional
information in \termpl*{attribute}\acroindex[!attribute]{SGML}. Additionally,
\acronym{SGML} documents may contain miscellaneous instructions for the programs
that are processing them as well as human-readable comments. An umbrella term for
the various parts of \acroshort{SGML} document is
\termpl*{node}\acroindex[!node]{SGML}. Repeated strings of text can be declared
as \termpl*[entities]{entity} \acroindex[!entity]{SGML} that can be used
throughout the document in place of the original strings. 

Although the described structure is shared by all \acronym{SGML} documents, the
actual syntax, as well as the restrictions regarding the contents and the
attributes of individual elements, are declared within a \acronym{DTD}, which
can be different for each document. It is worth noting that a \acronym{DTD} only
declares the syntax of an \acroshort{SGML} document; the semantics of the
individual elements and their attributes are left to the interpretation of the
program processing the document. The syntax and the constraints imposed by a
\acronym{DTD} define an \term*{application of \acronym{SGML}}%
\acroindex[!application]{SGML}. An \acronym{SGML} document is considered to be a
valid instance of an \acroshort{SGML} application, when it conforms to the
corresponding \acronym{DTD}.

\subsection{The Extensible Markup Language}
Although \acronym{SGML} was designed to be the general format for data exchange,
the complexity of the specification and the lack of support for Unicode (see
Section~\ref{sec:ucs+unicode}) proved to be a major hindrance preventing its
wider adoption and the development of \acroshort{SGML} tools. In a response,
\acronym{W3C} published a specification of \acronym{XML} in \citeyear{bray98}.
Along with the introduction of \acronym{XML}, the \acroshort{SGML} specification
received a technical corrigendum~\cite{goldfarb97:webSGML}, which turned
\acronym{XML} into an \acronym{SGML} application defined through a
\acronym{DTD}.

\begin{figure}
  \inputcode[xml]{examples/02/recipe.xml}
  \caption{An example \acronym{XML} document\filenamecap{recipe.xml}}
  \label{fig:recipe}\bigskip
\end{figure}

\begin{figure}
  \fsidenote{\acronym{DTD}s in \acronym{SGML} and \acronym{XML} documents can be
    either linked to the document through \code[dtd]{PUBLIC} and
    \code[dtd]{SYSTEM} identifiers (top), directly embedded in the document
    (middle), linked to the document and then extended by an embedded
    specification (bottom), or omitted.}
  \inputcode[dtd]{examples/02/dtdtypes}
  \caption{An example \acronym{DTD}}
  \label{fig:recipe-dtd}
\end{figure}
        
\begin{figure}
  \inputcode{examples/02/recipe.rnc}
  \caption{A reformulation of the \acronym{DTD} from Figure \ref{fig:recipe-dtd}
    in the compact syntax of the \acronym{Relax NG} schema
    language\filenamecap{recipe.rnc}. Note how \acronym{Relax NG} allows us to
    constrain the attribute data types.}
  \label{fig:recipe-rnc}
\end{figure}

This \acronym{DTD} completely fixes the syntax of \acronym{XML} documents, which
makes it possible to differentiate between two levels of correctness. An
\acronym{XML} document is considered to be \term*{well-formed}%
\acroindex[!well-formedness]{XML}, when it conforms to the \acronym{DTD} that
specifies the syntax of \acronym{XML} and to the \acroshort{XML} specification.
An \acroshort{XML} document is considered to be
\term*{valid}\acroindex[!validity]{XML} against an \acronym{DTD}, when it is
well-formed and conforms to the said \acronym{DTD}. Along with \acronym{DTD}s,
there exists a wealth of \termpl*{schema language}\acroindex[!schema
language]{XML}\sidenote{
  A list of tools for the manipulation of files in \acronym{XML} schema
  languages is maintained on the Web site of \acronym{W3C} at
  \url{http://www.w3.org/XML/Schema}.
} for \acronym{XML}---such as \acroshort{W3C} \acroshort{XML} Schema%
\acroindex[!Schema]{XML}, \acronym{Relax NG}, or Schematron---that can be used
to check the validity of an \acroshort{XML} document instead of a \acronym{DTD}.
The constrains imposed by either a \acronym{DTD} or a schema define an
\term*{application of \acronym{XML}} (also \term*{language} or \term*{format}).
\acroindex[!application]{XML} \acroindex[!language]{XML}
\acroindex[!format]{XML}

\begin{figure}
  \inputcode[xml]{examples/02/recipe.xsd}
  \caption{A reformulation of the \acronym{DTD} from Figure \ref{fig:recipe-dtd}
    in the \acroshort{XML} Schema \acroindex[!Schema]{XML}
    language\filenamecap{recipe.xsd}}
  \label{fig:recipe-xsd}
\end{figure}

\begin{figure}
  \inputcode[sh]{examples/02/recipe.sh}
  \caption{\acronym{XML} documents can be easily validated against \acronym{XML}
    schemata using the free command-line program of \cliutil{xmllint}.}
\end{figure}

Along with schema languages, other supplementary languages exist, such as
\inx{XPointer}, \inx{XPath}, and \inx{XQuery} for the retrieval of data from XML
documents, \acronym{CSS} for the specification of \acroshort{XML} document
design, and the various languages for the description of Web resources that we
will discuss in Section~\ref{sec:semantic-web}.

\begin{figure}[!b]
  {\tikzstyle{level 1}=[sibling distance=\baselineskip, level distance=1.5cm]
\begin{tikzpicture}[grow=right]
  \node {\textcolor{red}{Speech}}
    child {
      node [label=right:{AASE: See, you dare not! Every word of it's a
        lie!} ] {}}
    child {
      node[label=right:{PEER: Swear? Why should I?}] {} }
    child {
      node[label=right:{AASE: Well then, swear to me it's true!}] {}}
    child {
      node[label=right:{PEER: No, I'm not!}] {} }
    child {
      node[label=right:{AASE: Peer, you're lying!}] {} };
  \node [below=5\baselineskip] {\textcolor{blue}{Verse}}
    child {
      node[label=right:{Every word of it's a lie!} ] {}}
    child {
      node[label=right:{Swear? Why should I? See, you dare not!}] {} }
    child {
      node[label=right:{Well then, swear to me it's true!}] {}}
    child {
      node[label=right:{Peer, you're lying! No, I'm not!}] {} };
\end{tikzpicture}}%
\begin{Verbatim}[commandchars=\\\{\},codes={\catcode`$=3\catcode`^=7\catcode`_=8}]
<(\textcolor{blue}{V})line>
  <(\textcolor{red}{S})speech who="Aase">Peer, you're lying!</(\textcolor{red}{S})speech>
  <(\textcolor{red}{S})speech who="Peer">No, I'm not!</(\textcolor{red}{S})speech>
</(\textcolor{blue}{V})line><(\textcolor{blue}{V})line>
  <(\textcolor{red}{S})speech who="Aase">Well then,
    swear to me it's true!</(\textcolor{red}{S})speech>
</(\textcolor{blue}{V})line><(\textcolor{blue}{V})line>
  <(\textcolor{red}{S})speech who="Peer">Swear, why should I?</(\textcolor{red}{S})speech>
  <(\textcolor{red}{S})speech who="Aase">See, you dare not!
</(\textcolor{blue}{V})line><(\textcolor{blue}{V})line>
  Every word of it's a lie!</(\textcolor{red}{S})speech>
</(\textcolor{blue}{V})line>
\end{Verbatim}
%
  \caption{The markup of the dramatic and metrical views of \person{Henrik
    Ibsen}'s \work{Peer Gynt} using the \identifier{CONCUR} feature of
    \acronym{SGML}. This figure was inspired by the figures found in the article
    \protect\citework{sperberg04}.}
\end{figure}

A notable feature of \acronym{XML} unavailable in \acronym{SGML} are
\termpl*{namespace}\acroindex[!namespace]{XML}, which were added to the
\acroshort{XML} specification~\cite{bray99} in \citeyear{bray99}. Namespaces
enable the inclusion of elements and attributes from different \acronym{XML}
applications within a single \acronym{XML} document; each application is
uniquely identified through an \acropl{IRI}. Namespaces in \acronym{XML} are a
spiritual successor of a more expressive \acronym{SGML} feature of
\identifier{CONCUR}\index{CONCUR@\identifier{CONCUR}}, which makes it possible
to mark up several structural views of a single document. Unlike with
\identifier{CONCUR}, which ties each view to an \acroshort{SGML} \acronym{DTD},
there exists no general mechanism for the translation of the \acropl{IRI} to
\acronym{XML} schemata. This makes it impossible to validate namespaced
\acronym{XML} documents, unless all the \acropl{IRI} and their schemata are
known to the parser.

%%% Živoucí CONCAT <http://webylon.info/K.24>
%%% Popis SGML deklarace pro ISO/IEC15445
%%%   <http://www.angelovic.cz/internet/sgml-deklarace.html#concur>

Due to the reduced complexity of \acronym{XML} compared to \acronym{SGML}, the
language was adopted by the industry and has superseded \acronym{SGML} in most
applications. Some of the applications of \acronym{XML} for document preparation
include DocBook\acroindex[!DocBook]{XML}\sidenote{
  The authoritative resource on the DocBook \acronym{XML} format is
  \citework{walsh10}. The book itself is written in DocBook and its source code
  is publicly available at \url{http://docbook.org}.
}---a technical documentation markup language used for authoring books by
publishers such as O'Reilly Media and for documenting software at companies such
as Red Hat, \acroflat{SuSE}, or Sun Microsystems---, \acronym{TEI}---a general
text encoding markup language for the use in the academic field of digital
humanities---, \acronym{MathML}---a markup language for the description of
mathematical formulae---, or \acronym{SVG}---a vector graphics format. Other
\acroshort{XML} applications, such as \acroshort{XHTML} and
\acroshort{RDF}/\acroshort{XML}, will be discussed in
Section~\ref{sec:www-markup}.
      
\section{Markup on the World Wide Web}\label{sec:www-markup}
\subsection{The Hypertext Markup Language}
In \citeyear{bernerslee89}, an English computer scientist named \person{Timothy
John Berners-Lee} proposed a decentralized system for sharing documents within
\acronym{CERN}~\cite{bernerslee89}. The system laid foundation for the Web and
earned its author knighthood. The markup language used to write documents for
the system was an application of \acronym{SGML} called \acronym{HTML}. In 1993,
the Web started to gain traction among the general public owing largely to the
release of the first graphical Web browser Mosaic, which paved way for the Web
browsers of today. In 1994, \person{Timothy John Berners-Lee} formed
\acronym{W3C}, which has since developed the standards for the Web.

The first standard version of \acronym{HTML} was \acronym{HTML} 2.0
\cite{bernerslee95} published in \citeyear{bernerslee95}. As the Web was
becoming ubiquitous, it began accumulating an increasing number of documents
that weren't valid instances of \acronym{HTML}, since most Web browsers faced
with a malformed document would act in accordance with the Postel's law%
\sidenote{
  The Postel's law states that one should be conservative in what they send, but
  liberal in what they accept.~\cite[sec.\,2.10]{postel80} It is one of the base
  principles for building robust communication protocols.}
and try to render the document despite its deficiencies. In an attempt to unify
the way malformed \acronym{HTML} documents were rendered across the Web
browsers, \acronym{W3C} acknowledged and documented this behavior as a part of
the \acroshort{HTML}5 specification~\cite[sec.\,8.2]{hickson14}. An example of
a non-conforming \acroshort{HTML}5 document and its canonical interpretation is
given in Figure \ref{fig:overlapping-elements}.

\begin{figure}[b]
  \inputcode[html]{examples/02/malformed.html}
  \caption{The first line contains overlapping elements and, as such, can't be a
    part of a valid \acronym{HTML} document. Nevertheless, browsers should
    handle it identically to the second line.}
  \label{fig:overlapping-elements}
\end{figure}

Initially, \acronym{HTML} only comprised a mixture of
logical\index{markup!logical} and presentation markup\index{markup!presentation}
with fixed visual interpretation. This changed with the specification of
\acronym{CSS}, which was introduced by \acronym{W3C} in \citeyear{lie96}. The
language enabled the specification of the visual properties for any
\acroshort{HTML} element, which enabled the separation of document markup and
design, effectively eliminating the need for the presentation markup.

\begin{figure}
  \inputcode[html]{examples/02/presentation-markup.html}%
  \separatorcaption{An excerpt from the Web site of the \acroshort{CSS} Zen
    Zarden located at \protect\url{http://csszengarden.com}. The document above
    was created using the \acroshort{HTML} presentation
    markup\index{markup!presentation}. The document below achieves the same
    appearance by the combination of logical markup\index{markup!logical} and
    \acronym{CSS}.}
  \inputcode[html]{examples/02/logical-markup.html}
\end{figure}

During the same period, an initial version of a scripting language called
\inx{JavaScript}~\cite{ecma97} was drafted and incorporated into Netscape
Navigator 2.0\sidenote{
  \inx{JScript} and \inx{VBScript} competed directly with JavaScript, but they
  never saw implementation outside Microsoft browsers.
}---one of the contemporary leading web browsers and a descendant of the
original Mosaic browser. As a part of a joint effort by Sun Microsystems and
Netscape Communications to bring the programming language of Java into web
browsers, JavaScript was supposed to complement Java
applets~\cite{netscape95}---a role it has since outgrown. Standardized in
\citeyear{ecma97}~\cite{ecma97}, JavaScript blurred the line between static
documents and interactive applications and remains the predominant client-side
programming language of the Web. However, since the support of JavaScript by a
Web browser is fully optional, it is considered a good practice not to depend on
JavaScript for the rendering of \acroshort{HTML} documents. In the case of
interactive \acroshort{HTML} applications, this recommendation may be relaxed.

\subsection{The Extensible Hypertext Markup Language}
Ever since the release of \acronym{XML} in 1998, \acronym{W3C} entertained the
idea of turning \acronym{HTML} into an application of \acronym{XML}, rather than
of \acronym{SGML}, as exemplified by the working draft of \citework{raggett98}.
Unlike \acronym{HTML} parsers, whose acceptance of malformed content makes them
complex, \acronym{XML} parsers are required to strictly refuse \acronym{XML}
documents that aren't well-formed~\cite[Section~1.2, Terminology]{bray98},
leading to architectural simplicity and decreased computational requirements. As
a result, reformulating \acronym{HTML} in \acronym{XML} was suggested as a way
to bring the Web to mobile, embedded, and other devices limited in their
computational resources and to reduce the amount of malformed documents on the
Web in general. Other perceived advantages included the ability to use
\acronym{XML} tools for web documents and to include instances of other
\acronym{XML} applications---such as \acronym{MathML} and
\acronym{SVG}---directly into web documents through \acronym{XML} namespaces.

The idea was brought to fruition in the \acronym{XML} application of
\acronym{XHTML}. However, the supposed benefits proved to be too marginal to
warrant migration from \acronym{HTML}. The speed advantages of the simplified
processing were largely offset by the lack of support for incremental rendering,
since it is impossible to validate and render partially downloaded
\acroshort{XHTML} documents and the advances in the area of mobile devices made
\acronym{HTML} processing sufficiently fast. The lack of ways to provide
alternative content for browsers that would not support the \acroshort{XML}
applications instantiated in the \acroshort{XHTML} documents also reduced the
usefulness of the \acroshort{XML} namespaces in \acronym{XHTML} considerably. As
a result, \acronym{XHTML} has yet to succeed in replacing \acronym{HTML} and
remains a minority markup language on the Web.

%%% Content-Negotiation Techniques to serve XHTML as text/html and
%%% application/xhtml+xml
%%%   <http://www.w3.org/2003/01/xhtml-mimetype/content-negotiation>

\subsection{The Semantic Web and Linked Data}\label{sec:semantic-web}
The Web is based on the idea of a distributed and globally available network of
human knowledge. The languages of \acronym{HTML}, \acronym{XHTML}, \acronym{CSS}
and JavaScript form the foundation of the human-readable parts of the Web, but
are inadequate for creating a network of machine-readable data that could be
navigated by software agents.\sidenote{
  The idea of a network of machine-readable data was described by
  \person{Tim Berners-Lee} in \citeyear{bernerslee06} in the article
  \citework{bernerslee06}.}
Drawing from the research in the field of knowledge representation,
\acronym{W3C} created \acronym{RDF} in \citeyear{lassira99}---a language for the
description of resources on the Web.

An \acroshort{RDF} document represents data as a set of \termpl*{triplet}%
\acroindex[!triplet]{RDF}. Each triplet comprises a
\term*{predicate}\acroindex[!predicate]{RDF}, a \term*{subject}%
\acroindex[!subject]{RDF}, and an \term*{object}\acroindex[!object]{RDF},
where both the predicate and the subject are specified as \termpl*{resource}
\acroindex[!resource]{RDF} using \acropl{IRI}. If the object of a triplet
$(p,s,o)$ is also a resource, the triplet can be interpreted as a subject $s$
being in a relation $p$ with the object $o$. If the object is a \term*{literal
value} \acroindex[!literal]{RDF} rather than a resource, the triplet can be
interpreted as a subject $s$ having a property $p$ with the value $o$.

Resources in \acronym{RDF} are specified via \acropl{IRI} to prevent naming
collisions in \acronym{RDF} documents created independently by distinct authors.
These \acropl{IRI} do not need to point to any existing web page, and---beside
the small set of standard resources specified within the \acroshort{RDF}
specification---they carry no inherent meaning. In order to describe a set of
resources, the relationships between them, and their intended meaning in an
\acroshort{RDF} document, an extension of the set of standard resources called
\acroshort{RDF} Schema~\cite{brickley04} can be used. The resulting documents
are called \termpl*[ontologies]{ontology} \acroindex[!ontology]{RDF} and can be
used for automated reasoning about \acronym{RDF} documents containing resources
described by the ontology.\sidenote{
  A list of ontologies that are fully documented, honor the current best
  practices, and are supported by various tools can be found on the
  \acroshort{W3C} wiki at \url{http://www.w3.org/wiki/Good_Ontologies}.
} Some of the well-known ontologies include \acronym{DC}---an ontology for the
generic description of resources, both digital and physical---,
\acronym{FOAF}---an ontology for the description of people and their social
relationships---, or the Music Ontology---an ontology for the description of
entities related to the music industry, such as albums, artists, tracks, and
events. More expressive standards for the creation of ontologies, such as
\acronym{OWL}, also exist.

\acronym{RDF} documents can be represented through many languages, including
\acroshort{XML}~\cite{lassira99}, \acronym{JSON-LD},
\inx{Turtle}~\cite{beckett14:turtle}, and \inx{N-Triples}~\cite{beckett14:nt}.
Although \acroshort{RDF} documents in any of these representations can be
included in or linked to \acroshort{HTML} and \acroshort{XHTML} documents, this
will often result in the undesirable duplication of data. To prevent this, the
language of \acronym{RDFa} makes it possible to mark parts of the
\acroshort{HTML} or \acroshort{XHTML} document as \acronym{RDF} data. The usage
of \acronym{RDF} in conjunction with \acronym{HTML} and \acronym{XHTML} is
intended to gradually obsolete the loosely-defined use of \acroshort{HTML} and
\acroshort{XHTML} attributes, the \element{meta} and \element{link} elements,
and the \acronym{CSS} class names to include additional machine-readable metadata
into the documents on the Web---a technique known as \term{microformatting}.

\begin{figure}
  \inputcode[xml]{examples/02/john.rd}\vspace{-.4em}%
  \inputcode{examples/02/john.nt}\vspace{-.4em}%
  \inputcode{examples/02/john.ttl}
  \caption{An example \acronym{RDF} document using
    the \acronym{DC} and \acronym{FOAF} ontologies in the languages of
    \acronym{RDF}/\acronym{XML}\filenamecap[top]{john.rd},
    \inx{N-Triples}\filenamecap[middle]{john.nt}, and
    \inx{Turtle}\filenamecap[bottom]{john.ttl}}\label{fig:rdf-doc}
\end{figure}

\begin{figure}
  \inputcode[html]{examples/02/john.html.linked-rdf}
  \separatorcaption{Above is an \acroshort{HTML} document linked to the
    \acroshort{RDF} document from Figure \ref{fig:rdf-doc}. Below is the same
    \acroshort{HTML} document with the \acroshort{RDF} data directly embedded
    using the \acroshort{RDFa} language.}
  \inputcode[html]{examples/02/john.html.rdfa}
\end{figure}

\begin{figure}
  {\tikzstyle{level 1}=[sibling distance=5.5\baselineskip, level distance=0.5cm]
\tikzstyle{level 2}=[sibling distance=5.5\baselineskip, level distance=0.5cm]
\centerline{\begin{tikzpicture}[grow=right,->]
  \node[label=left:{\url{http://example.org/document.html}}] {}
    child { node {\code{"John's Web page"@en}}
      edge from parent
      node[left] {\textcolor{blue}{dc:title}}}
    child { node {\url{http://example.org/john-smith}}
      child { node[label=right:{foaf:Person}] {}
        edge from parent
        node[right,align=left] {\textcolor{blue}{rdf:type}}}
      child { node {\code{"John Smith"}}
        edge from parent
        node[left] {\textcolor{blue}{foaf:name}}}
      edge from parent
      node[left] {\textcolor{blue}{foaf:creator}};
    };
\end{tikzpicture}}}

  \caption{A graph of the \acroshort{RDF} document in Figure \ref{fig:rdf-doc}}
\end{figure}

%%% Linked Data
%%%   <http://www.w3.org/DesignIssues/LinkedData.html>
%%%
%%% Tim Berners-Lee: The next web
%%%   <http://www.ted.com/talks/tim_berners_lee_on_the_next_web>
%%%
%%% The SPARQL Query Language for RDF (SPARQL) and the (at the time of
%%% writing only drafted) Linked Data Fragments RDF query interfaces
%%%   <http://www.w3.org/TR/rdf-sparql-query/>
%%%   <http://linkeddatafragments.org/specification/>
%%%
%%% The Rule Interchange Format (RIF)
%%%   <http://www.w3.org/TR/rif-overview/>
        
\section{Document Preparation Systems}
Some of the existing markup languages are tied directly to specific
\acropl{DPS}. These \acropl{DPS} can be categorized into the
\term*{batch-oriented}\acroindex[!batch-oriented]{DPS}, which process text files
into printable output documents on demand, and the
\term*{interactive}\acroindex[!interactive]{DPS} (also \acronym{WYSIWYG}), which
allow the user to directly edit an approximation of the output document through
a visual editor. The price for the mild learning curve of interactive
\acropl{DPS} are the more primitive typesetting algorithms, which need to be
sufficiently fast to enable real-time user interaction, and the reduced
flexibility stemming from the usage of a \acronym{GUI}, which, although often
intuitive for simple tasks, seldom matches the power of the markup languages
used by batch-oriented \acropl{DPS}.

\subsection{Batch-oriented Systems}
One of the archetypal batch-oriented \acropl{DPS} are \cliutil*{troff}%
\index{troff@\cliutil*{troff}}, whose function is to produce output for general
printers, and \cliutil*{nroff}%
\index[see{\cliutil*{troff}}]{nroff@\cliutil*{nroff}}, whose function is to
produce output for line printers and text terminals. Both are proprietary
software developed for the Unix operating system at the beginning of 1970s by
\acronym{ATnT}. An alternative to \cliutil*{nroff} and \cliutil*{troff} is
\cliutil*{groff}\index[see{\cliutil*{troff}}]{groff@\cliutil*{groff}}, which was
developed as free software for the \acronym{GNU} project in 1980 by the members
of the \acronym{FSM}. \Cliutil*{groff} combines the capabilities of both systems
and is used extensively for the markup of documentation in \Unices. The
markup language of \cliutil*{groff} combines presentation
markup\index{markup!presentation} with programming constructs and enables the
definition of logical markup\index{markup!logical} through user macros. The
standard macro packages for \cliutil*{groff} include \identifier{man} for the
\index{troff@\cliutil*{troff}!\identifier{man}} formatting of documentation,
\identifier{me} \index{troff@\cliutil*{troff}!\identifier{me}} for the creation
of research papers, and the more recent \identifier{mom}
\index{troff@\cliutil*{troff}!\identifier{mom}} for general typesetting tasks.
Special markup invokes preprocessors that can be used for the typesetting of
tables, equations, and vector graphics.

\begin{figure}
  \vspace{.45em}%
  \fbox{\includegraphics[clip,trim=1.7cm 12.6cm 1.7cm 1.3cm,%
    width=0.975\textwidth]{examples/02/poe-groff.pdf}}%
  \vspace{.45em}%
  \inputcode[groff]{examples/02/poe.groff}
  \caption{An excerpt from the beginning of \person{Edgar Allen Poe}'s
    \work*{Cask of Amontillado}\workindex{the Cask of Amontillado} as a
    text marked up using the \identifier{mom} macro package of \cliutil*{groff}
    (below) and the output document (above). The marked up text was borrowed
    from the web page of \identifier{mom}~\cite{schaffter15}.}
  \label{fig:poe}
\end{figure}

%%% The Groff and Friends HOWTO
%%%   <http://troff.org/TheGroffFriendsHowto.pdf>
%%%
%%% Writing Macros
%%%   <https://www.gnu.org/software/groff/manual/html_node/Writing-Macros.html>
%%%
%%% A User's Guide to the Lout Document Formatting System
%%%   <https://www.urz.uni-heidelberg.de/imperia/md/content/urz/programme/text/lout.pdf>

Another notable free batch-oriented \acronym{DPS} is
\TeX\index{TeX@\TeX}\sidenote{
  The circumstances that led to the creation of \TeX\ and the surrounding tools
  are thoroughly documented in \citework{knuth98}.
}, which was developed in the 1970s by an American professor of computer science
\person{Donald Knuth} after he had received galley proofs for the second volume
of his monograph, \work{the Art of Computer Programming}, and found the
appearance of mathematical formulae distasteful. As a result, the typesetting of
mathematics is a central theme in \TeX, rather than an afterthought, which
differentiates it from most other \acropl{DPS} and which contributes to the
massive popularity \TeX\ has enjoyed among academics. Much like in the case of
\cliutil*{troff} and its derivatives, the language of \TeX\ contains only
typographic and programming primitives, but the creation of logical
markup\index{markup!logical} is possible through user macros. A popular \TeX\
macro package that enables the creation of various types of documents with just
logical markup is \LaTeX\index{LaTeX@\LaTeX}: the standard markup language for
academic and technical documents.

\begin{figure}
  \inputcode[tex]{examples/02/poe.tex}
  \caption{The document from Figure \ref{fig:poe} reformulated in
    \TeX\index{TeX@\TeX} using
    \hologo{plainTeX}\index{plainTeX@\hologo{plainTeX}} macros and the
    primitives of \hologo{eTeX}\index{eTeX@\hologo{eTeX}} and
    \hologo{pdfTeX}\index{pdfTeX@\hologo{pdfTeX}}}
\end{figure}

\subsection{Interactive Systems}
Interactive \acropl{DPS} come in two distinct flavors. Word processors
\acroindex[!interactive!word processing]{DPS} are the digital progeny of the
typewriter machine, whose output documents served as manuscripts to be typeset
by a typographer. With the advent of personal computing and the Web,
self-publishing became more affordable to the general public and modern word
processors can be used not only to write but also to design and typeset
documents, although the offered functionally is typically limited to ensure ease
of use. This concern is not shared by \term*{\acronym{DTP}
software}\acroindex[!interactive!desktop publishing]{DPS}, which provides
refined control over the resulting page layout and the typesetting at the
expense of a steeper learning curve.

Most interactive \acropl{DPS} will provide a means to mark up sections of text.
Presentation markup\index{markup!presentation} enables direct changes to the
design, whereas logical markup\index{markup!logical} enables the classification
of sections of text with the ability to set up the design of each class later
on. This decouples writing and markup from design and makes it easy to
consistently change the design of an entire document.

\begin{figure}
  \includegraphics[width=\textwidth]{examples/02/interactive-editors.png}
  \caption{Logical markup in the interactive \acropl{DPS} of \inx{Scribus}
    (left), \inx{Microsoft Word} (top), \inx{Adobe InDesign} (bottom left) and
    \inx{Apache OpenOffice} (bottom right)}
\end{figure}

\section{Lightweight Markup Languages}
Parallel to the heavy-duty applications of \acronym{SGML} and \acronym{XML},
there runs a vein of markup languages that give priority to unobtrusiveness and
legibility over raw expressive power. Rooted in the reality of computer text
terminals with limited formatting capabilities, \termpl{lightweight markup
language} leverage punctuation and indentation to produce comparatively weak and
domain-specific, but also humane, highly intuitive, and often profoundly
beautiful markup that is easy to both read and write. Examples of lightweight
markup languages include \inx{Markdown}, \inx{Creole}, \inx{AsciiDoc},
\inx{MakeDoc}, \inx{Setext}, and \inx{Wikicode}.  Lightweight markup languages
are typically supplemented by tools that enable the conversion to more general
markup languages, such as \acronym{HTML}. The more popular lightweight markup
languages come in various flavors that represent their use cases.

\chapter{Design}
After a manuscript has been written and marked up, it is time to create a visual
system that will emphasize the internal structure and the character of the
document. In book design, this involves the selection of one or several fonts
that are well-suited to both the document and each other, the design and the
positioning of marked-up elements---such as headings, tables, figures, and
lists---and parts of the page---such as headers, footers, and page numbers---,
and the choice of the paper size and the type area geometry. In web design,
several visual systems may have to be created to accommodate for various display
devices.

\section{Fonts}
When choosing fonts for a document, legibility should be of foremost concern.
The main text should be set with a font at a size of at least 10\,pt, if the
document is aimed at adult readers, or 12\,pt, if elementary-school children and
people with weakened sight are part of the audience
\cite[para.\,13--15]{kapr99}. The target medium also needs to be taken into
consideration. A faithful copy of a typeface designed for the letterpress will
look lighter than originally intended when printed digitally. This may hamper
its legibility, if it contains hairline strokes due to its high contrast or
overall lightness \cite[sec.~6.1.2]{bringhurst92}. In printed text, serified
fonts are more familiar to the reader and therefore more suitable for
long-distance reading than their sans serif counterparts. At small-resolution
screens, however, simple low-contrast typefaces with slab or no serifs will
often yield the best result.

Beside being legible, a font should be sympathetic to the subject matter and
compatible with the cultural and historical affiliations of the document. As
\person{Robert Bringhurst} writes in \citework*()[sec.\,6.3.1]{bringhurst92}:
\quote{There are books about the Renaissance set in faces that belong to the
Baroque and books about the Baroque set in faces from the Renaissance.  To a
good typographer, it is not enough merely to avoid these kinds of laughable
contradictions. The typographer seeks to \emph{shed light} on the text, to
generate insight and energy, by setting every text in a face and form in which
it actually belongs.} In a more nuanced way, this also applies to modern
documents with seemingly no historical affiliations. It is typical to set
scientific documents in neoclassical typefaces as an homage to the Age of
Enlightenment, and modern day novels in renaissance typefaces as a reference to
\person{Miguel de Cervantes Saavedra}'s \work{Don Quixote} and other early
European novels.

A font should also contain all the letters and symbols that will appear in
the document. If the manuscript is multilingual and contains passages in both
Latin and non-Latin writing systems, it may be necessary to combine several
typefaces. If the multilingual manuscript only contains Latin characters, but
several diacriticized characters are missing from the base font, they may be
constructed by combining the base font with diacritical marks from another
typeface. If certain punctuation marks and other symbols are missing from the
base font, they may likewise be borrowed from other typefaces. In each of these
cases, the typefaces should be consonant in their spirit and structure, unless
the text would benefit from their dissimilarity.

Several fonts may appear in a document---a bold font, an italic font, or perhaps
several sizes of the base font for use in some of the structural elements. The
natural instinct is to pick these fonts from one typeface, but some typefaces
may not offer all that the design requires. In that case, the fonts may again
have to be borrowed from other typefaces.

%%% unicode-range - CSS | MDN
%%%   <https://developer.mozilla.org/en-US/docs/Web/CSS/@font-face/unicode-range>
%%%
%%% Creating Custom Font Stacks with Unicode-Range
%%%   <https://24ways.org/2011/creating-custom-font-stacks-with-unicode-range>
%%%
%%% Fallback fonts on special characters
%%%   <http://stackoverflow.com/questions/11395584/fallback-fonts-on-special-characters>

\section{Structural Elements}
\subsection{Paragraphs}
The base units of thought in prose, paragraphs slice the text into coherent bits
ready to be consumed by the reader. A line in a paragraph of the main text
should be 45--75 characters long on a single-column page or 40--50 characters
long on a multi-column page and \term{justified} (spread horizontally to fit the
column width). Extended passages of lines wider than 80 characters strain the
eye of the reader, whereas justified lines that are too narrow to accommodate 40
characters may make the word spacing entirely too loose. In the latter case, the
text should be set \term{ragged} instead, as seen in the margin notes throughout
this book \cite[sec.\,2.1.2]{bringhurst92}.

% LaTeX, Word and the justification of paragraphs
% HTML, CSS and the justification and hyphenation of paragraphs

Vertically, the lines of a paragraph should be separated by approximately twenty
to forty-five percent of the font size \cite{line-spacing}. If the size of the
base font is 10\,pt, the line height (also known as the \emph{leading}) would be
between 12 and 14.5\,pt, adding 1 to 2.25\,pt of lead above and below each line.
As a general guideline, dark and bulky fonts require more leading, as do texts
riddled with accents, full capital letters, subscripts, and superscripts
\cite[sec.\,2.2.1]{bringhurst92}. The main text of this book is set in 10\,pt
Palatino with the leading of 12\,pt. To allow for such minimal leading, all
acronyms and other strings of upper-case letters are set as \term{small
capitals} (capital letters whose height matches the lower case).

% Leading in LaTeX, Word, and CSS

Two adjacent paragraphs should be visibly separated without distracting the
reader from the text. A predominant method is to indent the initial line of a
paragraph with one half (1\,en) to three times (3\,em) the font size
\cite[sec.\,2.3.2]{bringhurst92}. \par\noindent\hskip-\parindent If the margins
are ample, outdented paragraphs are an intriguing option as well. \ding{161}
Paragraphs can also be separated by graphical symbols, such as pilcrows, bullets,
or boxes.\hspace{3em}A plain horizontal space that is at least 3\,em wide can
likewise act as a paragraph separator
\cite[ch.\,2,~p.\,16]{beran14}.\par\noindent Block paragraphs exchange
indentation and horizontal separators for additional vertical space above and
below the paragraph. In justified block paragraphs, this space can be omitted as
well, although the typesetter then has to manually ensure that the last line of
each paragraph offers enough horizontal space to act as a separator. In short
documents and limited spans of text, block paragraphs are a breath of fresh air
\cite[sec.\,2.3.2]{bringhurst92}.

% Paragraph separation in LaTeX, Word, and CSS

The poetry counterpart to the paragraph, the stanza is a collection of lines
rather than sentences. Due to this structural difference, stanzas are typically
only justified, if the individual lines are long enough to fill up the column,
and ragged otherwise. Much like in the case of prose, short-form poetry benefits
from having the stanzas set in block paragraph style.

\subsection{Headings}
Another fundamental structural element, which appears in both prose and fiction,
is the heading, whose function is to delimit and name the individual sections of
a document. To alleviate navigation, headings should be a prominent presence on
a page. This can be achieved by using a larger variant of the base font or by
including the text of the latest heading in the margin or the header of the page
\cite[sec.\,4.2.1]{bringhurst92}, as seen throughout this book.

The hierarchy of the headings can be expressed by the variation of fonts,
indentation, alignment and numbering, although alternating the size of the base
font is sufficient for many types of a document. In documents that are bound in
codex form and read two pages at a time, the height of headings should be a
whole multiple of the line height of the main text, so that the headings do not
disrupt the alignment of lines on the facing pages \cite[para.\,33]{kapr99}.

% Headings in LaTeX, Word and CSS

\subsection{Tables}
\subsection{Margin Notes}

\section{Colors}
\section{Page Geometry}

\chapter{Typesetting}
\iffalse % This text is completely temporary
After finishing the design, the marked up manuscript can be flowed into the
document and checked for typesetting errors. Some of the more common typesetting
errors include \termpl{widow} and \termpl{orphan}---runt lines that are left
separated from the rest of the paragraph---, overfull lines, and improper line
breaks. In book design, only the actual errors found in the typeset document
need to be addressed. In web design, typographic errors need to be fixed
pre-emptively, as the document may display differently on various devices.
\fi
\section{Proofreading}

\chapter{Release}
\section{Printing and Binding}
\section{Distribution}


\backmatter

% Bibliography
\def\cite#1{$\!$}                    % Disable recursive citing.
\printbibliography[heading=bibintoc] % Print the bibliography.

% Acronyms
\fakechapter{Acronyms}
\def\index#1{} % Disable indexing of acronyms.
\printacronyms[heading=none]

% Index
\cleardoublepage
\def\index#1{} % Disable recursive indexing.
\printindex    % Print the index.
\end{document}
