\documentclass{book}
\usepackage{polyglossia}  % Hyphenation
\setmainlanguage{english}
\usepackage{minted}       % Source code typesettings
\setminted{breaklines=true}
\usepackage{placeins}     % Float barriers
\usepackage{tikz}         % Diagrams
\usetikzlibrary{trees,arrows}
\usepackage{fancyvrb}     % Manually colored verbatim
\usepackage{float}        % Non-floating figures
\usepackage{url}          % URLs
\usepackage[              % Bibliography
  backend=biber,
  style=numeric,
  citestyle=numeric-comp,
  sorting=none
]{biblatex}
\addbibresource{main.bib}
\usepackage{makeidx}      % Index
\makeindex
\usepackage[nottoc]{tocbibind}
\usepackage{./main}       % Markup and design
% Chapter 1
\defacronym{ASCII}{the American Standard Code for Information Interchange}
\defacronym{ASA}{the American Standard Association}
\defacronym{IBM}{the International Business Machines Corporation}
\defacronym{FORTRAN}{the FORmula TRANslator}
\defacronym{COBOL}{the COmmon Business-Oriented Language}
\defacronym{ISO}{the International Organization for Standardization}
\defacronym{IEC}{the International Electrotechnical Commission}
\defacronym{UCS}{the Universal multiple-octet coded Character Set}
\defacronym{BMP}{the Basic Multilingual Plane}
\defacronym{UTF}{the \acroshort{UCS} Transformation Format}
% Chapter 2
\defacronym{GML}{the General Markup Language}
\defacronym{SGML}{the Standard General Markup Language}
\defacronym{DTD}{Document Type Declaration}
\defacronym{XML}{the eXtensible Markup Language}
\defacronym{Relax NG}{the REgular LAnguage for \acronym{XML} New Generation}
\defacronym[cite=rfc3987]{IRI}{the Internationalized Resource Identifier}
\defacronym{CSS}{the Cascading Style Sheets language}
\defacronym{TEI}{the Text Encoding Initiative}
\defacronym{MathML}{the Mathematical Markup Language}
\defacronym{SVG}{the Scalable Vector Graphics language}
\defacronym[foreign={\foreign[french]{la Conseil Européen pour la Recherche
  Nucléaire}}]{CERN}{the European Organization for Nuclear Research}
\defacronym{HTML}{the HyperText Markup Language}
\defacronym{W3C}{the World Wide Web Consortium}
\defacronym{XHTML}{the eXtensible Hypertext Markup Language}
\defacronym{RDF}{the Resource Description Framework}
\defacronym{DC}{the Dublin Core}
\defacronym{FOAF}{Friend Or A Foe}
\defacronym{OWL}{the Web Ontology Language}
\defacronym{JSON-LD}{the JavaScript Object Notation for Linked Data}
\defacronym{RDFa}{\acronym{RDF} in attributes}
\defacronym{WYSIWYG}{What You See Is What You Get}
\defacronym{GUI}{Graphical User Interface}
\defacronym{GNU}{\acroflat{GNU} is Not Unix}
\defacronym{API}{Application Programming Interface}
\defacronym{ECMA}{the European Computer Manufacturers Association}
\mdefacronym{ATnT}{AT\scamp T}{the American Telephone and Telegraph corporation}
% Bibliography
\defacronym{USA}{the United States of America}
\defacronym{MA}{Massachusetts}
\defacronym{NY}{New York}
\defacronym{JTC}{the Joint Technical Committee}
\defacronym{SC}{a SubCommittee}
\defacronym{WG}{a Working Group}
\defacronym{RFC}{a Request For Comments}
        % Acronyms
\begin{document}

\frontmatter
\title{Electronic Document Preparation: An Author's Cookbook}
\author{Vít Novotný}
\maketitle
\tableofcontents
\mainmatter

\fakechapter{Foreword}
With the advent of the digital age, typesetting has become available to
virtually anyone equipped with a personal computer. Beautiful text documents can
now be crafted using free and consumer-grade software, which often obviates the
need for the involvement of a professional designer and typesetter. The level
playing field of the Internet coupled with the rising popularity of digital-only
documents then allows the author to bypass the publisher as well, if they so
wish, without jeopardizing their chance of recognition.

Documents are like tin-cans, conserving ideas for later consumption. This one in
particular contains a number of thoughts that should prove useful for any author
who aspires to write, design, typeset and distribute text documents---the most
universal and common kind of documents in use. Each chapters describes one
discrete step of text document preparation along with the relevant tools and
literature. Since document preparation is a broad, multidisciplinary subject,
the main goal of this text is to provide the reader with a general overview of
the landscape and to present references to more detailed literature for those
interested in further nourishment.

\chapter{Writing}
The essence of a document is the idea it represents. In the case of a text
document, this idea is articulated through speech, which is transcribed using
text and then laid out on a sheet of paper according to a design. And since the
text is typically independent on the design, whose task is to support and elicit
the internal structure of the text, it is writing that is the logical first step
in the text document creation.

  % An example of a poem by Christian Morgenstern
  % as an exception to the rule

The essentials of writing in any given natural language include \term{grammar
rules}, which specify the structure of spoken language, and \term{orthographic
rules}, which impose additional requirements on written text. The complexity of
either set of rules depends entirely on the language in question. Some
alphabets, such as the Japanese kanji, are not phonographic and the
correspondence between spoken words and written graphemes needs to be memorized
by the writer on a word-to-word basis. Some languages, such as Czech, use vastly
different grammar rules for speaking and for writing, which means that a spoken
sentence needs to be translated first before writing down. These specifics need
to be recognized by the writer.

On top of grammar and orthographic rules stand style guides, which, in order to
improve consistency, codify how common language patterns are encoded.\footnote{%
  This document was prepared in accordance with Strunk and White's
  \work{Elements of Style}: an American English style guide for general use.
} More comprehensive style guides---such as \work{the Chicago Manual of Style}
or Ritter's \work{Oxford Style Manual}---often go beyond writing and provide
guidelines on design and typesetting as well, making them an indispensable
reference on the editorial tradition.

Above all stand the \term{typographic rules}, which specify how the resulting
document should be laid out so that it doesn't disturb the eye of the reader.
These, as well as the orthographic rules on hyphenation, can be safely ignored
during writing, for the world of thought is unconstrained by the type area.

  % An illustration of the journey of an idea from the author to the reader

\section{Text processing}
\subsection{Text encoding}
\subsection{Text manipulation}
\subsection{Revision control}
\section{Word processing}

\chapter{Markup}
A manuscript can consist of a seamless river of words and still make perfect
sense to the author. To truly capture its meaning in a clear and unambiguous
manner, however, the manuscript will often need to be supplemented with a set of
annotations. At a more basic level, this refers to the compliance with the
orthographic rules---such as the correct spelling, hyphenation, capitalization,
word breaks, and punctuation---that are specific to the language of the document.
It is not at all unreasonable to expect that this basic compliance should be
already met by the manuscript. At a higher level, this consists of discovering
and marking up the inner order and logic of the text, so that the resulting
document can later be typeset in a way that visually reflects the structure of
the text.

\index{markup!logical|(}\index{markup!presentation|(}
To this end, there exists a wealth of \term{markup languages} that enable the
enrichment of text with additional information and labels. Aside from
\term{logical markup}, which enables the capture of the logical structure of the
document, markup languages may also provide \term{presentation markup}, which
directly impacts the visual properties of the document but carries no semantic
information. The usage of presentation markup instead of logical markup makes it
impossible to separate the markup from the design of the document and to capture
the logic of the text. As a result, the unity in the design of each logical
part of the document would have to be ensured manually, and future changes of
design would become tedious. It is for these reasons that the usage of
presentation markup is discouraged in favour of logical markup.
\index{markup!logical|)}\index{markup!presentation|)}

\section{Meta Markup Languages}
\subsection{The General Markup Language}
The situation engulfing digital typesetting was growing increasingly frustrating
for publishers in the 1960s. The markup languages used by different typesetting
systems varied wildly, and once a publisher had a large collection of documents
typeset via a given company, switching to another one could be very costly
venture. The companies would often take advantage of this situation, causing
their prices to skyrocket. As a result of that, a demand for a universal markup
language emerged.

This demand was met by a project developed\footnote{
  More information about the project can be found within the personal
  recollections of its co-author, Charles F. Goldfarb, in \cite{goldfarb96} and
  \cite{goldfarb97:whySGML}.
} at the Cambridge Scientific Center of \acronym{IBM} in the early 1970s. The
project aimed at imbuing a text editor with the ability to query, edit, and
display documents from a repository to allow the usage of computers in legal
practice. Very early on in the development process, it became clear that the
crux was going to be the markup languages in which the documents were written.
These languages were not unified and many of them comprised largely presentation
markup, which made information retrieval impossible without the use of
heuristics. To resolve the issue, a unifying markup language called
\acronym{GML} was drafted as a solution to the described problem. The language
was later released to the public in \cite{goldfarb81} and finally standardized
as \acronym{SGML}\footnote{
  The authoritative resource on \acronym{SGML} is \cite{goldfarb91}, which
  includes the full text of the standard along with the author's extensive
  annotations.
} within \cite{iso8879}.

\Acronym{SGML} documents consist of text mixed with \term{tags}
\acroindex[!tag]{SGML}, which delimit meaningful sections of the document called
\term{elements} \acroindex[!element]{SGML}. Elements can carry additional
information in \term{attributes} \acroindex[!attribute]{SGML}. Additionally,
\acronym{SGML} document can contain miscellaneous instructions for the program
that is processing it, as well as human-readable comments. Repeated strings of
text can be declared as \term{entities} \acroindex[!entity]{SGML} that can
consequently be used throughout the document in place of the original strings.

Although the described structure is shared by all \acronym{SGML} documents, the
actual syntax, as well as the restrictions with regards to the contents and the
attributes of individual elements, are declared within a \acronym{DTD}, which
can be different for each document. It is worth noting that a \acronym{DTD} only
declares the syntax of an \acroshort{SGML} document; the semantics of the
individual elements and their attributes are left to the interpretation of the
program processing the document. The syntax and the constraints imposed by a
\acronym{DTD} define an \term{application} \acroindex[!application]{SGML} of
\acronym{SGML}. An \acronym{SGML} document is considered to be a valid instance
of an \acroshort{SGML} application, when it conforms to the respective
\acronym{DTD}.

\subsection{The Extensible Markup Language}
Although \acronym{SGML} was designed to be the general format for data exchange,
the complexity of the specification and the lack of support for Unicode proved
to be a major hindrance preventing its wider adoption and tool development. As a
response, \acronym{W3C} published a specification of \acronym{XML} within
\cite{bray98} in 1998. Along with the introduction of \acronym{XML}, the
\acroshort{SGML} specification received a technical corrigendum of
\cite{goldfarb97:webSGML}, which turned \acronym{XML} into a proper subset of
\acronym{SGML} restrained by an \acroshort{SGML} \acronym{DTD}.

\begin{figure}
  \inputminted{xml}{examples/02/recipe.xml}
  \caption{An example \acronym{XML} document\filename{recipe.xml}}
  \label{fig:recipe}\bigskip
  \inputminted{dtd}{examples/02/dtdtypes}
  \caption{\acronym{SGML} and \acronym{XML} \acronym{DTD}s can be either linked
    to the document through public and system identifiers, directly embedded in
    the document, both linked and embedded, or left out altogether.}
  \label{fig:recipe-dtd}
\end{figure}
        
This \acronym{DTD} completely fixes the syntax of \acronym{XML} documents, which
makes it possible to differentiate two levels of correctness. Specifically, an
\acronym{XML} document is considered to be \term{well-formed}%
\acroindex[!well-formedness]{XML}, when it conforms to the \acroshort{SGML}
\acronym{DTD} that restrains \acronym{XML} as well as to the additional
constraints given in the specification. An \acroshort{XML} document is
considered to be \term{valid} \acroindex[!validity]{XML} against an
\acroshort{XML} \acronym{DTD}, when it is well-formed and conforms to the said
\acronym{XML} \acronym{DTD}.  Along with \acronym{DTD}s, there exists a wealth
of \term{schema languages}\acroindex[!schema language]{XML}\footnote{
  A list of tools for the manipulation of files in \acronym{XML} schema
  languages is maintained on the web site of \acronym{W3C} at
  \url{http://www.w3.org/XML/Schema}.
} for \acronym{XML}, such as \acroshort{W3C} \acroshort{XML} Schema
\acroindex[!Schema]{XML}, \acronym{Relax NG}, or Schematron that can be used to
check the validity of an \acroshort{XML} document instead of a \acronym{DTD}.
The constrains imposed by either a \acronym{DTD} or a schema define an
\term{application}, \term{language}, or \term{format}
\acroindex[!application]{XML} \acroindex[!language]{XML}
\acroindex[!format]{XML} of \acronym{XML}.

Along with schema languages, other supplementary languages also exist, such as
\inx{XPointer}, \inx{XPath}, and \inx{XQuery} for addressing sets of elements
(fragments) within a \acronym{XML} document or \acronym{CSS} for specifying the
visual properties of an \acroshort{XML} document. Although some of these
languages may not be \acronym{XML} languages, they are nevertheless used within
documents of various \acronym{XML} formats and form an important part of the
ecosystem.

A notable new feature of \acronym{XML} are \term{namespaces}%
\acroindex[!namespace]{XML}, which were added to the specification with the
release of \cite{bray99} in 1999. Namespaces enable the inclusion of elements
and attributes of different \acronym{XML} applications within a single
\acronym{XML} document by providing a method to qualify element and attribute
names with \acropl{IRI} that uniquely represent the respective \acronym{XML}
applications. Namespaced elements are a spiritual successor of a more expressive
\acronym{SGML} feature of \inx{\identifier{CONCUR}}, which makes it possible to
mark up several structural views of a single document. Unlike with
\identifier{CONCUR}, which ties each view to an \acroshort{SGML} \acronym{DTD},
there exists no general mechanism for the translation of \acropl{IRI} to
\acronym{XML} schemata.  This makes it impossible to validate namespaced
\acronym{XML} documents, unless all used \acropl{IRI} and their respective
schemata are known to the parser.

\begin{figure}[hb!]
  {\tikzstyle{level 1}=[sibling distance=\baselineskip, level distance=1.5cm]
\begin{tikzpicture}[grow=right]
  \node {\textcolor{red}{Speech}}
    child {
      node [label=right:{AASE: See, you dare not! Every word of it's a
        lie!} ] {}}
    child {
      node[label=right:{PEER: Swear? Why should I?}] {} }
    child {
      node[label=right:{AASE: Well then, swear to me it's true!}] {}}
    child {
      node[label=right:{PEER: No, I'm not!}] {} }
    child {
      node[label=right:{AASE: Peer, you're lying!}] {} };
  \node [below=5\baselineskip] {\textcolor{blue}{Verse}}
    child {
      node[label=right:{Every word of it's a lie!} ] {}}
    child {
      node[label=right:{Swear? Why should I? See, you dare not!}] {} }
    child {
      node[label=right:{Well then, swear to me it's true!}] {}}
    child {
      node[label=right:{Peer, you're lying! No, I'm not!}] {} };
\end{tikzpicture}}%
\begin{Verbatim}[commandchars=\\\{\},codes={\catcode`$=3\catcode`^=7\catcode`_=8}]
<(\textcolor{blue}{V})line>
  <(\textcolor{red}{S})speech who="Aase">Peer, you're lying!</(\textcolor{red}{S})speech>
  <(\textcolor{red}{S})speech who="Peer">No, I'm not!</(\textcolor{red}{S})speech>
</(\textcolor{blue}{V})line><(\textcolor{blue}{V})line>
  <(\textcolor{red}{S})speech who="Aase">Well then,
    swear to me it's true!</(\textcolor{red}{S})speech>
</(\textcolor{blue}{V})line><(\textcolor{blue}{V})line>
  <(\textcolor{red}{S})speech who="Peer">Swear, why should I?</(\textcolor{red}{S})speech>
  <(\textcolor{red}{S})speech who="Aase">See, you dare not!
</(\textcolor{blue}{V})line><(\textcolor{blue}{V})line>
  Every word of it's a lie!</(\textcolor{red}{S})speech>
</(\textcolor{blue}{V})line>
\end{Verbatim}

  \caption{The markup of the dramatic and metrical views of the beginning of
    Henrik Ibsen's \work{Peer Gynt} using the \identifier{CONCUR} feature of
    \acronym{SGML}}
\end{figure}

%%% Živoucí CONCAT <http://webylon.info/K.24>
%%% Popis SGML deklarace pro ISO/IEC15445
%%%   <http://www.angelovic.cz/internet/sgml-deklarace.html#concur>

\begin{figure}[H]
  \inputminted{xml}{examples/02/recipe.xsd}
  \caption{A reformulation of the recipe \acronym{DTD} from Figure
    \ref{fig:recipe-dtd} in the \acroshort{XML} Schema \acroindex[!Schema]{XML}
    language.\filename{recipe.xsd}}
  \label{fig:recipe-xsd}
  \inputminted{text}{examples/02/recipe.rnc}
  \caption{A reformulation of the recipe \acronym{DTD} from Figure
    \ref{fig:recipe-dtd} in the compact syntax of \acronym{Relax NG}.%
    \filename{recipe.rnc}}
  \label{fig:recipe-rnc}
  \inputminted{sh}{examples/02/recipe.sh}
  \caption{\acronym{XML} documents can be easily validated against \acronym{XML}
    schemata using command-line tools, such as \inx{\identifier{xmllint}}.}
\end{figure}

Due to the reduced complexity of \acronym{XML} compared to \acronym{SGML}, the
language was adopted by specialists and the general public alike and has
superseded \acronym{SGML} in many applications. Some of the applications of
\acronym{XML} for document preparation include DocBook\acroindex[!DocBook]{XML}%
\footnote{
  The authoritative resource on the DocBook \acronym{XML} format is
  \cite{walsh10}. The book itself is written in DocBook and its source code is
  publicly available at the Web page at \url{http://docbook.org}.
}---a technical documentation format used for authoring books by publishers such
as O'Reilly Media and for documenting software at companies such as Red Hat,
\acroflat{SuSE}, or Sun Microsystems---, \acronym{TEI}---a general text encoding
format for the use in the academic field of digital humanities---,
\acronym{MathML}---a format for describing mathematical formulae---, or
\acronym{SVG}---a two-dimensional vector image format. Other \acroshort{XML}
applications, such as \acroshort{XHTML} and \acroshort{RDF}/\acroshort{XML}, will
be discussed in Section \ref{sec:www-markup}.
      
\section{Markup on the World Wide Web}\label{sec:www-markup}
\subsection{The Hypertext Markup Language}
In 1989, Timothy John Berners-Lee proposed in \cite{bernerslee89} a
decentralized system for sharing linked documents within \acronym{CERN}. The
system laid foundation for today's \term{\inx{World Wide Web}} (Web) and earned
its author knighthood. The markup language used to write documents for the
system was an application of \acronym{SGML} called \acronym{HTML}. In 1993, the
Web started to gain popularity amongst the general public owing to the release
of the first graphical Web browser Mosaic, which paved way for the Web browsers
of today. In 1994, Timothy John Berners-Lee formed \acronym{W3C}, which has
since been developing the standards for the Web.

The first standard version of \acronym{HTML} was \acronym{HTML} 2.0 published as
\cite{rfc1866} in 1995. As the Web was becoming ubiquitous, it began
accumulating an increasing number of documents that weren't valid instances of
\acronym{HTML}, since most Web browsers, when faced with a malformed document,
would act in accordance with the Postel's law and try to render the document
despite its deficiencies. In an attempt to unify the way malformed
\acronym{HTML} documents were rendered across the Web browsers, \acronym{W3C}
acknowledged and documented this behaviour as a part of \cite[Section~8.2,
Parsing HTML documents]{hickson14} in 2008. An example is presented in Figure
\ref{fig:overlapping-elements}.

\begin{figure}[b]
  \begin{minted}[linenos]{html}
<b>This text is bold, <i>bold and italic</b>, italic.</i>
<b>This text is bold, </b><i><b>bold and italic</b>, italic.</i>
  \end{minted}
  \caption{The fragment on line 1 contains overlapping elements and, as such, it
    can't be a part of a valid \acronym{HTML} document. Nevertheless, it is
    recommended by \acronym{W3C} that browsers should parse the fragment
    identically to the fragment on line 2.}
  \label{fig:overlapping-elements}
\end{figure}

Initially, \acronym{HTML} comprised a mixture of logical and presentation markup
with fixed visual interpretation. In 1996, a specification of \acronym{CSS} was
published by \acronym{W3C} within \cite{lie96}. The language enabled the
specification of the visual properties of any element, which allowed for the
separation of document markup and design, effectively eliminating the need for
the presentation markup.

\begin{figure}
  \inputminted{html}{examples/02/presentation-markup.html}
  \caption{An excerpt from the Web site of the \acroshort{CSS} Zen Zarden
    located at \protect\url{http://csszengarden.com}. The document above was
    created using the \acroshort{HTML} presentation markup. The document below
    achieves the same appearance by the combination of logical markup and
    \acronym{CSS} definitions.}\bigskip
  \inputminted{html}{examples/02/logical-markup.html}
\end{figure}

During the same period, an initial version of a scripting language called
\inx{JavaScript} was drafted and incorporated into Netscape Navigator 2.0, one
of the contemporary leading web browsers and a descendant of the original Mosaic
browser. As a part of a joint effort to bring Java into web browsers by Sun
Microsystems and Netscape Communications, JavaScript was, according to
\cite{js-announcement}, supposed to complement Java applets -- a role it has
since outgrown. Standardized within \cite{ecma1} in 1997, JavaScript blurs the
line between static documents and interactive applications and remains the
predominant client-side programming language for the Web. Since, however, the
support of JavaScript by a Web browser is fully optional, it is considered a
good practice to use it chiefly for the enrichment of already self-sufficient
\acronym{HTML} documents. In case of an interactive application, this
recommendation can be relaxed.

\subsection{The Extensible Hypertext Markup Language}
Ever since the release of \acronym{XML} in 1998, \acronym{W3C} entertained the
idea of turning \acronym{HTML} into an application of \acronym{XML}, rather than
\acronym{SGML}, as exemplified by the working draft of \cite{raggett98} released
the very same year. Unlike \acronym{HTML} parsers, which are complex in their
acceptance of malformed content, \acronym{XML} parsers are, as per
\cite[Section~1.2, Terminology]{bray98}, required to draconianly refuse
\acronym{XML} documents that aren't well-formed, leading to architectural
simplicity and decreased computational requirements. As a result, reformulating
\acronym{HTML} in \acronym{XML} was suggested as a way to bring the Web to
mobile, embedded, and other devices limited in their resources, as well as to
reduce the amount of malformed documents on the Web in general. Other perceived
advantages included the ability to use \acronym{XML} tools for web documents and
to include instances of other \acronym{XML} applications, such as
\acronym{MathML} and \acronym{SVG}, directly into web documents using
\acronym{XML} namespaces.

The idea was brought to fruition within \cite{pemberton00} in 2000 as an
\acronym{XML} application named \acronym{XHTML}. \acronym{XHTML} was met with
lukewarm reception, since many of its supposed benefits proved to be either
questionable or too marginal to warrant migration from \acronym{HTML}. The
speed advantages of the simpler parser were largely offset by its lack of
support for incremental rendering, caused by the impossibility to validate
partially downloaded pages, the closing of the gap in the computing power
between mobile and desktop devices made it possible to use full-fledged
\acronym{HTML} parsers across the spectrum, and the lack of ways to provide
alternative content for browsers that would not support directly included
\acroshort{XML} documents considerably reduced the usefulness of \acronym{XML}
namespaces. As a result, \acronym{XHTML} has yet to succeed in replacing
\acronym{HTML} and remains an alternative markup language for the Web.

%%% Content-Negotiation Techniques to serve XHTML as text/html and
%%% application/xhtml+xml
%%%   <http://www.w3.org/2003/01/xhtml-mimetype/content-negotiation>

\subsection{The Semantic Web and Linked Data}
The underlying fundament of the Web is the the idea of a globally available and
incrementally scalable base of human knowledge. \acronym{HTML} and
\acronym{XHTML} succeeded in fulfilling this vision for human-readable documents
but didn't provide a unifying machine-readable format for the representation of
structured information that would enable the creation of a web of linked data
running in parallel to the web of documents. In 1999, \acronym{W3C} released
\cite{lassira99} containing the specification of \acronym{RDF}, a method for the
description of resources on the Web.

Drawing from the research in the field of knowledge representation, an
\acroshort{RDF} document represents data as a set of \term{triples}
\acroindex[!triple]{RDF}. Each of the triplets comprises \term{a
predicate}\acroindex[!predicate]{RDF}, \term{a
subject}\acroindex[!subject]{RDF}, and \term{an object}\acroindex[!object]{RDF},
where both the predicate and the subject are specified as \term{resources}
\acroindex[!resource]{RDF} using \acropl{IRI}. If the object of a triplet
$(p,s,o)$ is also a resource, the triplet can be interpreted as a subject $s$
being in a relation $p$ with the object $o$. If the object is a \term{literal
value} \acroindex[!literal]{RDF} rather than a resource, the triplet can be
interpreted as a subject $s$ having a property $p$ with the value $o$.

Resources in \acronym{RDF} are specified via \acropl{IRI} to prevent naming
collisions in \acronym{RDF} documents created independently by distinct authors.
These \acropl{IRI} are not required to resolve to an actual web page, and
disregarding the small set of standard resources specified within the
\acroshort{RDF} specification, they carry no inherent meaning. In order to
describe a set of resources, the relationships between them, and their intended
meaning in an \acroshort{RDF} document, an extension of the set of standard
resources---called \acroshort{RDF} Schema and specified within
\cite{brickley04}---can be used. The resulting documents are called
\term{ontologies} \acroindex[!ontology]{RDF} and can be used for automated
reasoning about \acronym{RDF} documents containing resources described by the
ontology.\footnote{
  A list of ontologies that are fully documented, honor the current best
  practices, and are supported by various tools can be found on the
  \acroshort{W3C} wiki at \url{http://www.w3.org/wiki/Good_Ontologies}.
} Some of the well-known ontologies include \acronym{DC}---an ontology for the
generic description of both web multimedia and physical objects---,
\acronym{FOAF}---an ontology for the description of people and their social
relationships---, or the Music Ontology---an ontology for the description of
entities related to the music industry, such as albums, artists, tracks, and
events. More expressive standards for the creation of ontologies, such as
\acronym{OWL} specified within \cite{mcguinness04}, also exist.

The syntax of \acronym{RDF} is not fixed, meaning that, beside the
\acroshort{XML} serialization specified within \cite{lassira99}, other
languages, such as \acronym{JSON-LD} specified within \cite{sporny14}, the
\inx{Turtle} language specified within \cite{beckett14:turtle}, or the
line-based \inx{N-Triples} language specified within \cite{beckett14:nt}, can be
used to represent an \acroshort{RDF} document. A noteworthy serialization of
\acronym{RDF} is \acronym{RDFa} specified within \cite{adida08}. Although
various serializations of \acronym{RDF} can be included in or linked to an
\acroshort{HTML} or \acroshort{XHTML} document, this will often result in an
undesirable duplication of data already present in the document. To prevent
this, \acronym{RDFa} utilizes the content already present within the underlying
\acroshort{HTML} or \acroshort{XHTML} document using element attributes. The
usage of \acronym{RDF} in conjunction with \acronym{HTML} and \acronym{XHTML} is
intended to gradually obsolete the practice of using the \element{meta} and
\element{link} elements to provide additional application-specific metadata about
the document.

\begin{figure}
  \inputminted{xml}{examples/02/john.rd}
  \inputminted{text}{examples/02/john.nt}
  \inputminted{text}{examples/02/john.ttl}
  \caption{Above is an example \acronym{RDF} document using \acronym{DC} and
    \acronym{FOAF} ontologies in \acronym{RDF}/\acronym{XML}%
    \filename{john.rd}, \inx{N-Triples}\filename{john.nt}, and \inx{Turtle}%
    \filename{john.ttl} serializations. Below is a graph representation of the
    document.}\label{fig:rdf-doc}\bigskip
  {\tikzstyle{level 1}=[sibling distance=5.5\baselineskip, level distance=0.5cm]
\tikzstyle{level 2}=[sibling distance=5.5\baselineskip, level distance=0.5cm]
\centerline{\begin{tikzpicture}[grow=right,->]
  \node[label=left:{\url{http://example.org/document.html}}] {}
    child { node {\code{"John's Web page"@en}}
      edge from parent
      node[left] {\textcolor{blue}{dc:title}}}
    child { node {\url{http://example.org/john-smith}}
      child { node[label=right:{foaf:Person}] {}
        edge from parent
        node[right,align=left] {\textcolor{blue}{rdf:type}}}
      child { node {\code{"John Smith"}}
        edge from parent
        node[left] {\textcolor{blue}{foaf:name}}}
      edge from parent
      node[left] {\textcolor{blue}{foaf:creator}};
    };
\end{tikzpicture}}}

\end{figure}

\begin{figure}[t!]
  \inputminted{html}{examples/02/john.html.linked-rdf}
  \caption{Above is an \acroshort{HTML} document linked to the \acroshort{RDF}
    document from Figure \ref{fig:rdf-doc}. Below is the same \acroshort{HTML}
    document with the \acroshort{RDF} data directly embedded using the
    \acroshort{RDFa} serialization.}\bigskip
  \inputminted{html}{examples/02/john.html.rdfa}
\end{figure}

%%% Tim Berners-Lee: The next web
%%%   <http://www.ted.com/talks/tim_berners_lee_on_the_next_web>
%%%
%%% The SPARQL Query Language for RDF (SPARQL) and the (at the time of
%%% writing only drafted) Linked Data Fragments RDF query interfaces
%%%   <http://www.w3.org/TR/rdf-sparql-query/>
%%%   <http://linkeddatafragments.org/specification/>
%%%
%%% The Rule Interchange Format (RIF)
%%%   <http://www.w3.org/TR/rif-overview/>
        
\section{Markup in Document Preparation Systems}
Some of the existing markup is directly tied with specific document preparation
systems. These can be dichotomized into \term{batch-oriented systems}
\index{document preparation system!batch-oriented}, which process marked up input
text documents into printable output documents in a page description language on
demand, and \term{interactive systems} \index{document preparation
system!interactive}, which allow the user to directly edit an approximation of
the output document via a visual editor. Also referred to as \acronym{WYSIWYG},
interactive systems exchange mild learning curve for more primitive typesetting
algorithms that don't stand in the way of interactivity and for reduced
flexibility stemming from the usage of a \acronym{GUI}, which, although often
intuitive for simple tasks, seldom matches the power of the markup languages
used by batch-oriented systems.

\subsection{Batch-oriented Document Preparation Systems}
One of the archetypal batch-oriented systems are \inx{\identifier{troff}}, whose
function is to produce output for general printers, and \identifier{nroff}%
\index{\identifier{nroff}|seealso{\identifier{troff}}}, whose function is to
produce output for line printers and text terminals. Both tools were developed
as a proprietary software for the Unix operating system at the beginning of
1970s by \acronym{ATnT}. An alternative to \identifier{nroff} and
\identifier{troff} is
\identifier{groff}\index{\identifier{groff}|seealso{\identifier{troff}}}, which
was developed as a free software in 1980 by \acronym{GNU}. \Identifier{groff}
combines the capabilities of both tools and is still used extensively for the
markup of documentation in Unix and Unix-like operating systems, although more
advanced alternatives for general typesetting (such as \TeX) exist. The
associated markup language combines presentation markup with programming
constructs and allows the definition of logical markup in the form of user
macros. Standard macro packages, such as \inx{\identifier{man}} for the
formatting of documentation, \identifier{me}
\index{\identifier{troff}!\identifier{me}} for the creation of research papers,
or the more recent \identifier{mom} \index{\identifier{troff}!\identifier{mom}}
for general typesetting tasks are typically distributed along with the system.
Special markup invokes preprocessors that can be used for the typesetting of
tables, equations, and vector graphics.

\begin{figure}
  \fbox{\includegraphics[clip,trim=1.7cm 12cm 1.7cm 1.3cm,%
    width=0.975\textwidth]{examples/02/poe-groff.pdf}}
  \inputminted{groff}{examples/02/poe.groff}
  \caption{An excerpt from the beginning of Edgar Allen Poe's \work{Cask of
    Amontillado} formatted using the \identifier{mom}
  \index{\identifier{troff}!\identifier{mom}} macro package of
  \inx{\identifier{groff}}.}
  \label{fig:poe}
\end{figure}

%%% The Groff and Friends HOWTO
%%%   <http://troff.org/TheGroffFriendsHowto.pdf>
%%%
%%% Writing Macros
%%%   <https://www.gnu.org/software/groff/manual/html_node/Writing-Macros.html>

Another notable batch-oriented system is \TeX\index{TeX@\TeX}\footnote{
  The circumstances that led to the creation of \TeX\ and the surrounding tools
  are thoroughly documented in \cite{knuth98}.
}, which was developed and consequently released to the public domain in 1970s
by a professor of computer science Donald Knuth, after he had received galley
proofs for the second volume of his monograph, \work{the Art of Computer
Programming}, and found the appearance of mathematical formulae distasteful.
As a result, the typesetting of mathematics is a central theme in \TeX, rather
than an afterthought, which differentiates it from most other document
preparation systems and which largely contributed to the massive popularity
\TeX\ has attained in academia. Much like in the case of \identifier{troff} and
its derivatives, the language of \TeX\ is ripe with typographic and programming
primitives and doesn't by itself contain logical markup. The language, however,
allows for the creation of user macros. An example of a macro package that makes
it possible to create documents of various classes with pure logical markup is
\inx{\LaTeX}: the standard markup language for academic and technical documents.

\begin{figure}
  \inputminted{tex}{examples/02/poe.tex}
  \caption{The \identifier{groff} document from Figure \ref{fig:poe}
    reformulated using \TeX}
\end{figure}

\subsection{Interactive Document Preparation Systems}
Interactive systems come in two distinctive flavours. \emph{Word processors}
\index{document preparation system!interactive!word procesing} are
the digital progeny of the typewriter machine, with which they share the main
function of fast text capture. The typewriter output documents served largely as
a manuscript for designers and typographers, who then produced the actual
published work. With the advent of personal computing and the Web,
self-publishing became more approachable to the general public, and modern word
processors can be used not only to write but also to design and typeset
documents, although the offered functionally is typically constrained by user
experience demands. This concern is not shared by \emph{desktop publishing
software}\index{document preparation system!interactive!desktop publishing},
which provides refined control over the resulting page layout and the
typesetting primitives. The ability to produce document comparable to those set
using traditional typography comes at the expense of steeper learning curve.

Most interactive systems will provide a means to mark up sections of text.
Presentation markup performs direct changes of typographic properties, such as
the face, family, variant, color, or the size of the font that is used to
typeset the text. Logical markup enables the classification of sections of text
with the ability to set up the design of each class later on. This decouples
writing and markup from design and makes it easy to change the design of the
entire document and difficult to produce inconsistent design.

\begin{figure}
  \includegraphics[width=\textwidth]{examples/02/interactive-editors.png}
  \caption{Logical markup in the interactive document preparation systems of
    \inx{Scribus} (left), \inx{Microsoft Word} (above), \inx{Adobe InDesign}
    (below left) and \inx{Apache OpenOffice} (below right)}
\end{figure}

\section{Lightweight Markup Languages}
Parallel to the heavy-duty applications of \acronym{SGML} and \acronym{XML},
there runs a vein of markup languages that give priority to unobtrusiveness and
legibility over expressiveness and genericity. Rooted in the reality of
computer text terminals with limited formatting capabilities, \term{lightweight
markup languages} \index{lightweight markup language} leverage punctuation and
indentation to produce comparatively weak and domain-specific, but also humane,
highly intuitive, and often profoundly beautiful markup that is easy to both
read and write. Examples of lightweight markup languages include \inx{Markdown},
\inx{Creole}, \inx{AsciiDoc}, \inx{MakeDoc}, \inx{Setext}, or \inx{Wikicode}.
Lightweight markup languages are typically supplemented by tools that enable
the conversion to more general languages such as \acronym{HTML}. Also typical
is the existence of numerous flavours of every lightweight markup language,
which represent different use cases.

\chapter{Design}
% XML -- CSS, XSL, XSLT
\chapter{Typesetting}
% Accessibility -- WCAG
\chapter{Proofreading}
\chapter{Printing}
\chapter{Distribution}

% Bibliography
\cleardoublepage
\printbibliography[heading=bibintoc]

% Acronyms
\cleardoublepage
\fakechapter{Acronyms}
\printacronyms[heading=none]

% Index
\printindex
\faketoc{Index}

\end{document}
