\chapter{Design}
After a manuscript has been written and marked up, it is time to create a visual
system that will emphasize the internal structure and the character of the
document. In book design, this involves the selection of one or several fonts
that are well-suited to both the document and each other, the design and the
positioning of marked-up elements---such as headings, tables, figures, and
lists---and parts of the page---such as headers, footers, and page numbers---,
and the choice of the paper size and the type area geometry. In web design,
several visual systems may have to be created to accommodate for various display
devices.

\section{Fonts}
When choosing fonts for a document, legibility should be of foremost concern.
The main text should be set with a font at a size of at least 10\,pt, if the
document is aimed at adult readers, or 12\,pt, if elementary-school children and
people with weakened sight are part of the audience
\cite[para.\,13--15]{kapr99}. The target medium also needs to be taken into
consideration. A faithful copy of a typeface designed for the letterpress will
look lighter than originally intended when printed digitally. This may hamper
its legibility, if it contains hairline strokes due to its high contrast or
overall lightness \cite[sec.~6.1.2]{bringhurst92}. In printed text, serified
fonts are more familiar to the reader and therefore more suitable for
long-distance reading than their sans serif counterparts. At small-resolution
screens, however, simple low-contrast typefaces with slab or no serifs will
often yield the best result.

Beside being legible, a font should be sympathetic to the subject matter and
compatible with the cultural and historical affiliations of the document. As
\person{Robert Bringhurst} writes in \citework*()[sec.\,6.3.1]{bringhurst92}:
\quote{There are books about the Renaissance set in faces that belong to the
Baroque and books about the Baroque set in faces from the Renaissance.  To a
good typographer, it is not enough merely to avoid these kinds of laughable
contradictions. The typographer seeks to \emph{shed light} on the text, to
generate insight and energy, by setting every text in a face and form in which
it actually belongs.} In a more nuanced way, this also applies to modern
documents with seemingly no historical affiliations. It is typical to set
scientific documents in neoclassical typefaces as an homage to the Age of
Enlightenment, and modern day novels in renaissance typefaces as a reference to
\person{Miguel de Cervantes Saavedra}'s \work{Don Quixote} and other early
European novels.

A font should also contain all the letters and symbols that will appear in
the document. If the manuscript is multilingual and contains passages in both
Latin and non-Latin writing systems, it may be necessary to combine several
typefaces. If the multilingual manuscript only contains Latin characters, but
several diacriticized characters are missing from the base font, they may be
constructed by combining the base font with diacritical marks from another
typeface. If certain punctuation marks and other symbols are missing from the
base font, they may likewise be borrowed from other typefaces. In each of these
cases, the typefaces should be consonant in their spirit and structure, unless
the text would benefit from their dissimilarity.

Several fonts may appear in a document---a bold font, an italic font, or perhaps
several sizes of the base font for use in some of the structural elements. The
natural instinct is to pick these fonts from one typeface, but some typefaces
may not offer all that the design requires. In that case, the fonts may again
have to be borrowed from other typefaces.

%%% unicode-range - CSS | MDN
%%%   <https://developer.mozilla.org/en-US/docs/Web/CSS/@font-face/unicode-range>
%%%
%%% Creating Custom Font Stacks with Unicode-Range
%%%   <https://24ways.org/2011/creating-custom-font-stacks-with-unicode-range>
%%%
%%% Fallback fonts on special characters
%%%   <http://stackoverflow.com/questions/11395584/fallback-fonts-on-special-characters>

\section{Structural Elements}
\subsection{Paragraphs}
The base units of thought in prose, paragraphs slice the text into coherent bits
ready to be consumed by the reader. A line in a paragraph of the main text
should be 45--75 characters long on a single-column page or 40--50 characters
long on a multi-column page and \term{justified} (spread horizontally to fit the
column width). Extended passages of lines wider than 80 characters strain the
eye of the reader, whereas justified lines that are too narrow to accommodate 40
characters may make the word spacing entirely too loose. In the latter case, the
text should be set \term{ragged} instead, as seen in the margin notes throughout
this book \cite[sec.\,2.1.2]{bringhurst92}.

% LaTeX, Word and the justification of paragraphs
% HTML, CSS and the justification and hyphenation of paragraphs

Vertically, the lines of a paragraph should be separated by approximately twenty
to forty-five percent of the font size \cite{line-spacing}. If the size of the
base font is 10\,pt, the line height (also known as the \emph{leading}) would be
between 12 and 14.5\,pt, adding 1 to 2.25\,pt of lead above and below each line.
As a general guideline, dark and bulky fonts require more leading, as do texts
riddled with accents, full capital letters, subscripts, and superscripts
\cite[sec.\,2.2.1]{bringhurst92}. The main text of this book is set in 10\,pt
Palatino with the leading of 12\,pt. To allow for such minimal leading, all
acronyms and other strings of upper-case letters are set as \term{small
capitals} (capital letters whose height matches the lower case).

% Leading in LaTeX, Word, and CSS

Two adjacent paragraphs should be visibly separated without distracting the
reader from the text. A predominant method is to indent the initial line of a
paragraph with one half (1\,en) to three times (3\,em) the font size
\cite[sec.\,2.3.2]{bringhurst92}. \par\noindent\hskip-\parindent If the margins
are ample, outdented paragraphs are an intriguing option as well. \ding{161}
Paragraphs can also be separated by graphical symbols, such as pilcrows, bullets,
or boxes.\hspace{3em}A plain horizontal space that is at least 3\,em wide can
likewise act as a paragraph separator
\cite[ch.\,2,~p.\,16]{beran14}.\par\noindent Block paragraphs exchange
indentation and horizontal separators for additional vertical space above and
below the paragraph. In justified block paragraphs, this space can be omitted as
well, although the typesetter then has to manually ensure that the last line of
each paragraph offers enough horizontal space to act as a separator. In short
documents and limited spans of text, block paragraphs are a breath of fresh air
\cite[sec.\,2.3.2]{bringhurst92}.

% Paragraph separation in LaTeX, Word, and CSS

The poetry counterpart to the paragraph, the stanza is a collection of lines
rather than sentences. Due to this structural difference, stanzas are typically
only justified, if the individual lines are long enough to fill up the column,
and ragged otherwise. Much like in the case of prose, short-form poetry benefits
from having the stanzas set in block paragraph style.

\subsection{Headings}
Another fundamental structural element, which appears in both prose and fiction,
is the heading, whose function is to delimit and name the individual sections of
a document. To alleviate navigation, headings should be a prominent presence on
a page. This can be achieved by using a larger variant of the base font or by
including the text of the latest heading in the margin or the header of the page
\cite[sec.\,4.2.1]{bringhurst92}, as seen throughout this book.

The hierarchy of the headings can be expressed by the variation of fonts,
indentation, alignment and numbering, although alternating the size of the base
font is sufficient for many types of a document. In documents that are bound in
codex form and read two pages at a time, the height of headings should be a
whole multiple of the line height of the main text, so that the headings do not
disrupt the alignment of lines on the facing pages \cite[para.\,33]{kapr99}.

% Headings in LaTeX, Word and CSS

\subsection{Tables}
\subsection{Margin Notes}

\section{Colors}
\section{Page Geometry}
