\chapter{Writing}
The essence of a document is the idea it represents. In the case of a text
document, this idea is articulated through speech, which is transcribed using
text, optionally accompanied by figures, and then laid out on a sheet of paper
according to a design. Since the text is typically independent on the design,
whose task is to support and elicit the internal structure of the text, it is
writing that is the logical first step in the text document creation.

\begin{figure}
  \input examples/01/trichter
  \caption{Exceptions that prove the rule about the separation of text and
    design can sometimes be encountered in poetry. Above is \person{Christian
    Morgenstern}'s \foreign[german]{\work{Trichter}}, where the text and its
    form are intimately intertwined.}
\end{figure}

The essentials of writing in any given natural language include \termpl*{grammar
rule}\index{writing rules!grammar}, which specify the structure of spoken
language, and \termpl*{orthographic rule}\index{writing rules!ortography}, which
impose additional requirements on written text. The complexity of either set of
rules depends entirely on the language in question. Some writing systems, such
as the Japanese kanji, are not phonographic and the correspondence between
the spoken words and the written symbols needs to be memorized by the writer on
a word-to-word basis. Other languages may use vastly different grammar rules for
speaking and for writing, which means that a spoken sentence needs to be
translated first before writing down. A writer needs to recognize these
specifics.

On top of grammar and orthographic rules stand \termpl{style guide}, which, in
order to improve consistency, codify how common language patterns are encoded.
More comprehensive style guides---such as \work{the Chicago Manual of Style} or
\work{the Oxford Style Manual}\footnote{
  This document was prepared in accordance with \person{William Strunk}'s
  \work{Elements of Style}, an American English style guide for general use. 
}---often go beyond writing and provide guidelines on design and typesetting as
well, making them an indispensable reference on the editorial tradition.

Above all stand the \termpl*{typographic rule}\index{writing rules!typography},
which specify how the resulting document should be typeset so that it doesn't
disturb the eye of the reader. These, as well as the orthographic rules on
hyphenation, can be left out of consideration during writing, as it is the page
that should be formed around the writing and not the other way around.

\section{Text Processing}
Originally the domain of the pen, the quill, the stylus, and the more recent
typewriter machine, manuscripts of today are produced chiefly using the personal
computer and stored in \termpl{text file}. The discipline of creating and
manipulating digital text is called \term{text processing} and will be the focus
of this section.

\subsection{Character Encoding}
Although computing at its most primal has no use for anything but numbers, it
has nevertheless been accompanied by text from the very outset. Even the
earliest computers from 1950s were programmed with both raw machine code and
the text programming languages of \acronym{FORTRAN} and \acronym{COBOL}. The
digital representation of letters, digits and other characters was initially
closely tied to each specific application and processor architecture, but with
the advent of networking in 1960s, mutual intelligibility became a point of
concern.\iffalse\footnote{
  \Acronym{EBCDIC} by \acronym{IBM} was the default encoding on
  \acroshort{IBM}'s System/360 mainframes and was in active use until the
  introduction of \acronym{PC} in 1981. In countries using Chinese ideographs,
  special encodings, such as Big5, \acronym{JIS}, and \acronym{EUC}, are used to
  this day. For brevity, the text focuses on the main stream of international
  encodings.}\fi
\ \quote{We had over sixty different ways to represent characters in
computers. It was a real Tower of Babel,} explains \cite{brandel99}
\person{Bob Berner}, an American computer scientist who worked at \acronym{IBM}
during 1956--1962 and who drafted \acronym{ASCII}---a
\term{character encoding} from \citeyear{asa63} that unified the industry and
enabled computer networking on large scale.

\subsubsection{ASCII}
In \acronym{ASCII}, every character is represented by a number from zero to 127,
which is transformed to a seven-bit integer called a \term{character code}.
These 128 codes are used to encode \termpl{printable character}---spanning the
letters of the English alphabet, digits, punctuation, and other symbols---and
\termpl{control code}, as depicted in Table \ref{tab:ascii}.  Unlike printable
characters, control codes have no fixed visual representation and they were used
to implement application-specific communication protocols and text formatting;
their precise semantics were defined in the much later standard of
\acroshort{ISO}~646 from \citeyear{iso72} \cite{iso72}. Unconstrained by the
bandwidth and the storage limitations of the 1960s and 1970s, today's
communication protocols and text formats gravitate towards markup constructed
from printable characters, which, unlike control codes, are easy to read and
write by humans.

\begin{table}
  \input examples/01/ascii
  \caption{The \acronym{ASCII} encoding, as specified in the \citeyear{asa86}
    revision of the standard \cite{asa86}.}
  \label{tab:ascii}
\end{table}

\begin{table}
  \input examples/01/utf8
  \caption{The \acroshort{UTF}-8 encoding. Each \textvisiblespace\ represents
    one bit of the \acroshort{UCS} code point in binary.}
  \label{tab:utf8}
\end{table}

\begin{table}
  \input examples/01/utf8-example
  \caption{An example of the \acroshort{UTF}-8 encoding}
  \label{tab:utf8-example}
\end{table}

%%% ASCII, Other Standards
%%%   <https://en.wikipedia.org/wiki/ASCII#Other_standards>
%%%   <https://en.wikipedia.org/wiki/ISO/IEC_8859-2#External_links>
%%%
%%% Character histories: notes on some Ascii code positions
%%%   <http://www.cs.tut.fi/~jkorpela/latin1/ascii-hist.html>
%%%
%%% ASCII: American Standard Code for Information Infiltration
%%%   <http://worldpowersystems.com/J/codes/>
%%%
%%% Encyclopedia of Computer Science, 4th edition
%%%   <https://dl.acm.org/ralston.cfm?CFID=698031209&CFTOKEN=46500462>
%%%
%%% Theoretical Foundation of Regular Expressions and Text Editors
%%%   <http://citeseerx.ist.psu.edu/viewdoc/download?rep=rep1&type=pdf&doi=10.1.1.126.9920>
%%%
%%% A History of Scientific Text Processing at CERN
%%%   <http://ref.web.cern.ch/ref/CERN/CNL/2001/001/tp_history/>
%%%
%%% When and how did text enter the world of computing, eventually to be
%%% standardized as ASCII in 1963? <http://qr.ae/RHFEzE>
%%%
%%% IBM's Early Computers: A Technical History
%%%   <http://www.amazon.com/IBMs-Early-Computers-Technical-Computing/dp/0262523930>

The following properties make it easy to manipulate and reason about character
strings encoded in \acronym{ASCII}:
\begin{itemize}
  \item Each character is represented by exactly seven bits. This makes it easy
    to allocate space for character strings of fixed length, to measure the
    number of characters stored in a memory region, and to perform basic
    operations, such as adjacent character retrieval or text truncation.
  \item Characters are alphabetically ordered. Character strings can therefore
    be collated by comparing character code binary values.
  \item Lowercase and uppercase letters, digits and control codes form
    contiguous ranges of character codes, simplifying classification.
  \item There is precisely one way to encode any printable character. The
    conversion between the lower- and uppercase letters is a matter of
    inverting one bit.
\end{itemize}
This comes at the expense of support for non-English writing systems. As a
temporary workaround, a set of \acroshort{ASCII} derivatives that replaced the
less-needed characters of \# \$ @ [ \textbackslash\ ] \textasciicircum\ ` \{ |
\} and \textasciitilde\ for international characters was specified in the
\acroshort{ISO}~646 standard \cite{iso72} from \citeyear{iso72}.

\subsubsection{Eight-bit Encodings}
With the byte size stabilizing at eight bits, new character encodings emerged
that were based on \acronym{ASCII} and used the additional bit to encode
characters of non-English writing systems while retaining complete backwards
compatibility with \acroshort{ASCII}. Beside the numerous vendor-specific
encodings (called \termpl{code page}), a set of 15 eight-bit encodings covering
all major modern writing systems whose characters fit within the space of 128
additional combinations was standardized in the
\acroshort{ISO}/\acroshort{IEC}~8859 series released during 1986--2001.

  % Show a time diagram of Czech encodings
  %   <http://luki.sdf-eu.org/txt/cs-encodings-faq.html>

Compared to \acronym{ASCII}, eight-bit encodings introduced an additional level
of complexity to text processing:
\begin{itemize}
  \item Each character is exactly eight bits wide. The manipulation with strings
    is therefore as straightforward as with \acronym{ASCII}.
  \item Character strings can no longer be collated by character code
    comparison. Each encoding requires a separate mapping from character codes
    to sorting weights.
  \item Classes of characters, such as uppercase and lowercase letters or
    punctuation, no longer form contiguous ranges and their position varies
    among encodings. This impedes character classification.
  \item Idiosyncrasies, such as the ligature of æ and invisible hyphenation
    hints, are included in several encodings, which makes it more difficult to
    determine character string equivalence. Algorithms for case conversion vary
    among encodings.
  \item There exists no standard mechanism to detect which encoding is being
    used. The distinction needs to be done on the application level using either
    heuristics or additional metadata. Consequently, no standard mechanism
    exists to use different character encodings within a single text document.
\end{itemize}
A portion of this complexity is inherent in the task of encoding the characters
of all modern writing systems, but the overhead caused by the character encoding
fragmentation proved to be unnecessary.

\subsubsection{The Universal Character Set and Unicode}
In the early 1990s, the continual increase in the available bandwidth and
storage led to the creation of the standards of Unicode
\cite{unicode91,unicode92} and \acronym{UCS} in an attempt to create a text
encoding that would contain the characters of all the world's living languages
and succeed \acronym{ASCII} as the \foreign[italian]{lingua franca} of text
interchange.

\Acronym{UCS} is an ever-expanding catalogue of characters from writing systems
both modern and ancient, and symbols ranging from diacritical marks,
punctuation, and ideograms to mahjong tiles, alchemical symbols, and the ancient
Greek musical notation. Each of these characters is assigned a number, called
a \term{code point}, ranging from 0 to 2,147,483,647 (\hexa{7FFFFFFF}) with the
numbers of the most common characters in the range from 0 to 65,535
(\hexa{FFFF}) called \acronym{BMP}. The smallest unit of division in
\acronym{UCS} are \termpl*{block}\acroindex[!block]{UCS}, which contain 256
thematically related characters. \Acronym{UCS} encodings map character numbers
to binary character codes.

Three major encodings\footnote{
  Notable are also the seven-bit encodings of \acroshort{UTF}-7
  \acroindex[!UTF-7@\acroshort{UTF}-7]{UTF} and \inx{Punycode}, which bring
  Unicode support to protocols that were designed with the seven-bit
  \acroshort{ASCII} in mind, such as e-mail.}
are specified in the \acronym{UCS} standard and its amendments
\cite{iso93:am1,iso93:am2}:
\begin{description}
  \item[\acroshort{UTF}-32]\acroindex[!UTF-32@{\acroshort{UTF}-32}]{UTF}Directly
    encodes \acronym{UCS} characters by transforming their code points to
    four-byte integers. \acroshort{UTF}-32 is also known as
    \acroshort{UCS}-4\acroindex[!UCS-4@{\acroshort{UCS}-4}]{UCS}.
  \item[\acroshort{UTF}-16]\acroindex[!UTF-16@{\acroshort{UTF}-16}]{UTF}
    Directly encodes characters within \acronym{BMP} by transforming their code
    points to two-byte integers. Code points in the range from 65,536 to
    1,114,111 (\mbox{\hexa{010000}--\hexa{10FFFF}}) are transformed into pairs
    of two-byte integers, called \termpl{surrogate pair}, ranging from
    55,296 to 57,343 (\mbox{\hexa{DC00}--\hexa{DFFF}}). To enable the
    \acroshort{UTF}-16 encoding, the code points in the surrogate pair range
    will never be assigned \cite[sec.\,3.4]{unicode15}. The same applies to code
    points greater than 1,114,111 (\hexa{10FFFF}), which allows
    \acroshort{UTF}-16 to encode any \acroshort{UCS} character.
  \item[\acroshort{UTF}-8]\acroindex[!UTF-8@{\acroshort{UTF}-8}]{UTF}
    Directly transforms code points ranging from 0 to 127 (\hexa{7F}) to
    one-byte integers. Since the first \acroshort{UCS} block of the
    \acroshort{BMP} matches \acronym{ASCII}, any text encoded in eight-bit
    \acroshort{ASCII} is also encoded in \acroshort{UTF}-8. Code points in the
    range from 127 to 1,114,111 (\mbox{\hexa{00007F}--\hexa{10FFFF}}) are
    transformed into two to four one-byte integers ranging from 128 to 253
    (\mbox{\hexa{80}--\hexa{FD}}). The encoding is illustrated in tables
    \ref{tab:utf8} and \ref{tab:utf8-example}.
\end{description}
\acroshort{UTF}-32 is primarily used for the internal representation of
individual \acronym{UCS} characters inside programs, \acroshort{UTF}-16 fulfills
a similar role in applications that only work with \acronym{BMP}, and
\acroshort{UTF}-8 is used for text storage and interchange. Since 2010, the
majority of text content on the Web is encoded in \acroshort{ASCII} and
\acroshort{UTF}-8 \cite{qsuccess15}.

Originally a competing standard, Unicode underwent a merger with \acronym{UCS}
in version 1.1 and since then, the standards have been kept closely
synchronised. Unicode is a superset of \acronym{UCS}, which defines additional
information about \acronym{UCS} characters---such as their general category,
directionality, case, or numeric value \cite[sec.\,3.5 and ch.\,4]{unicode15}%
---, various text processing algorithms, and implementation guidelines.

\begin{figure}
  \ffootnote{One of the design goals of \acroshort{UCS} was to avoid assigning
    code points to different glyphs that carry the same meaning. As a result,
    the visually distinctive Han characters used in the East Asian countries
    of China, Japan, Korea, and Vietnam were merged into a set of 75,960
    ideograms in a process referred to as the \term{Han Unification}. This
    simplifies text processing, but also makes it impossible to encode a text
    in multiple East Asian languages without having to rely on external markup
    to select appropriate regional fonts. As a result, a derivative of
    \acronym{UCS} that doesn't implement the Han Unification was developed for
    use in operating systems based on \acronym{TRON} and is used in the East
    Asia alongside \acronym{UCS} and region-specific encodings.}
  \input examples/01/unihan
  \caption{Several Han characters in the traditional Chinese, Japanese,
    Korean, and Vietnamese variants}
\end{figure}

Regarding text processing, Unicode and \acronym{UCS} represent a compromise
between the simplicity of the seven-bit \acronym{ASCII} and the heterogeneity of
eight-bit encodings:
\begin{itemize}
  \item If simple text manipulation is preferred over space efficiency, each
    character can be made exactly two or four bytes wide using the
    \acroshort{UTF}-16 and \acroshort{UTF}-32 encodings.
  \item Although character strings can not be collated by a simple character
    code comparison, a collation algorithm is defined in the Unicode
    specification \cite{unicode15:collation} and collation tables for major
    locales \cite{unicode15:cldr} are maintained by the Unicode Consortium.
  \item Classes of characters---such as uppercase letters, lowercase letters,
    numbers, and punctuation---do not form contiguous ranges, but their position
    is directly specified by the standard \cite[sec.\,4.5]{unicode15}.
  \item Although idiosyncrasies---such as ligatures, invisible hyphenation
    hints, and combining characters---are present in \acronym{UCS}, explicit
    normalization algorithms for character string equivalence testing are
    specified by the standard \cite[sec.\,2.12]{unicode15}. An algorithm
    for case conversion is also specified \cite[sec.\,3.13]{unicode15}.
    \index{Unicode!normalization}\index{Unicode!case conversion}
  \item The \ucs[FEFF]{Byte Order Mark} character can be inserted at the
    beginning of a text as a signature of Unicode encodings. As the name
    suggests, the order in which the \hexa{FE} and \hexa{FF} bytes arrive also
    indicates the order of bytes (called \term{endianity}) that was used to
    encode integers. In \acroshort{UTF}-32 and \acroshort{UTF}-16, endianity
    can be chosen arbitrarily by the encoding application. In \acroshort{UTF}-8,
    one-byte integers are used and the notion of endianity is therefore
    meaningless.
\end{itemize}

\begin{figure}
  \input examples/01/combining-chars
  \caption{Some \acroshort{UCS} characters can be either input as a single
    entity or composed from several combining characters. Regarding Unicode
    normalization forms, all of the above representations are canonically
    equivalent.}
\end{figure}

\begin{figure}
  \centerline{\code[sh]{iconv -f latin2 -t utf8 -- old.txt > new.txt}}%
  \caption{Text files can be converted between encodings using the
    \cliutil{iconv} command-line tool. The sample code shows the file
    \filename{old.txt} being converted from the
    \acroshort{ISO}/\acroshort{IEC}~8859-2 encoding to \mbox{\acroshort{UTF}-8}.
    The result of the conversion is stored in the file \filename{new.txt}.}
\end{figure}

%%% Setting collation order in shell sort
%%%   <http://superuser.com/a/414408/136765>
%%%
%%% Unicode 101: An Introduction to the Unicode Standard
%%%   <http://www.interproinc.com/blog/unicode-101-introduction-unicode-standard>
%%%
%%% Unicode Implementation levels
%%%   <http://www.cl.cam.ac.uk/~mgk25/unicode.html#levels>
%%%
%%% Unification of the Unicode Standard and ISO 10646
%%%   <http://www.unicode.org/versions/Unicode1.0.0/V2ch01.pdf>
%%%
%%% Unicode 88 <http://unicode.org/history/unicode88.pdf>
%%% Unicode equivalence <https://en.wikipedia.org/wiki/Unicode_equivalence>
%%%
%%% Plane (Unicode):
%%%   <https://en.wikipedia.org/wiki/Plane_(Unicode)#Basic_Multilingual_Plane>

\begin{figure}[p]
  \centerline{\includegraphics[width=0.75\textwidth]%
    {examples/01/google-pinyin.png}}
  \caption{Text input methods are not limited to keyboard layouts. Software that
    allows for the input of non-Latin characters on a keyboard through reversed
    romanization can often be the best option for writing systems with a large
    number of characters. Above is the \inx{Google Pinyin} input method for the
    Android operating system, which makes it possible to input Chinese
    characters using the \inx{pinyin} phonetic system.}
\end{figure}

\begin{figure}[p]
  \input examples/01/composeKey
  \caption{The \key{Compose}\index{Compose@\displaykey{Compose}} key followed by
    a mnemonic sequence of \acroshort{ASCII} characters produces a
    \acroshort{UCS} character. Although originally a physical key, \key{Compose}
    is not available on modern \acroshort{PC} and Apple keyboards and is usually
    mapped to a less-used key, such as the right \key{Ctrl} or \key{Super} key.
    \key{Compose} is natively supported on \Unices\ using the \inx{X Window
    System}. On other operating systems, support can be added by third-party
    software.}
\end{figure}

\begin{figure}
  \input examples/01/altCodes
  \caption{On the Windows operating system, holding the \key{Alt} key and typing
    a sequence of numbers produces a character with the corresponding number
    from either an \acroshort{IBM} code page, if the number has no leading zero,
    or from a Windows code page otherwise. The code pages vary depending on the
    current locale; in English locales, the \acroshort{IBM} code page~437
    and the Windows code page~1252 are used. After a Windows Registry
    modification, it is also possible to directly produce \acroshort{UCS}
    characters by holding the \key{Alt} key and typing the corresponding
    \acroshort{UCS} code point in hexadecimal.}
\end{figure}

\subsection{Text Input}
To insert text into a document, it is necessary to use an input device. In case
of personal computers, this is typically a computer keyboard and a mouse,
although the ongoing research in the areas of \acronym{SR} and \acronym{OCR}
makes it possible to use a microphone or a tablet as well. On hand-held devices,
the use of either a numeric keypad or a touch-screen is more typical.

An operating system will typically provide one or more input methods for each
input device through a component commonly referred to as the \acronym{IME}. The
\acroshort{ASCII} encoding was developed with typewriters and teleprinters in
mind and, as their direct descendant, the standard computer keyboard provides
support for all \acroshort{ASCII} characters. This doesn't apply to the much
larger \acronym{UCS} and it is the task of an \acronym{IME} to provide a
mechanism for the creation and selection of keyboard layouts that will allow the
user to input any \acronym{UCS} character. Some programs may provide input
methods of their own.

\subsection{Text Editors}
A \term{text editor} is an applications, which can be used to create and modify
text files. Entry-level text editors are often distributed with an operating
system and offer little beyond the ability to load, modify and save text files
in a text encoding of choice. Entry-level text editors with a \acronym{GUI}
include the \inx{Windows Notepad}, the iOS \inx{TextEdit} in plain text mode,
and \inx{Leafpad} for \Linux\ and \acronym{BSD}. Entry-level text editors with a
\acronym{CLI} include \cliutil{joe}, \acroshort{GNU}
\cliutil*{nano}\acroindex[!nano]{GNU}, and \cliutil{pico}.

More advanced text editors come with the support for \termpl{regular expression}
and \term{version control}---which will be covered in sections \ref{sec:regexs}
and \ref{sec:vcs}---, as well as user modules that extend the base
functionality.  Advanced \acronym{GUI} text editors include \inx{Sublime Text},
\inx{Atom}, and \inx{PSPad}. Advanced \acronym{CLI} text editors include
\cliutil{emacs}, \cliutil{vi}, and \cliutil{vim}. The presented \acronym{CLI}
text editors are especially notorious for their learning curve, whose steepness
is only matched by the power these editors grant to those who wield them.

\subsection{Interactive Document Preparation Systems}
Interactive \acropl{DPS}\acroindex[!interactive]{DPS} are a breed of text
editors that produces fully-formatted text documents instead of (or along with)
text files. The reader is advices to avoid interactive \acropl{DPS} that use
proprietary, undocumented, and obscure file formats which lock the user into
using the respective \acronym{DPS} to open the files reliably. Well-defined
interactive \acronym{DPS} file formats include \acronym{PDF}, \acronym{OOXML},
and \acronym{ODF}.

The primary difference between text editors and \acropl{DPS} is the fact that
the user is expected to use the \acronym{DPS} to mark up, design, and typeset the
resulting text document, whereas with plain text files a multitude of choices is
available at each step of the document preparation process. The self-sufficient
nature of \acropl{DPS} may be a time-saving feature for simpler documents, but
in the case of more complex documents, the markup and typesetting capabilities
of a \acronym{DPS} may not be up to par with those of a dedicated tool.
Interactive \acropl{DPS} include \inx{Apache OpenOffice}, \inx{TextEdit},
\inx{Microsoft Word}, \inx{Scribus}, \inx{Adobe InDesign}, \inx{Adobe
FrameMaker} and \inx{QuarkXPress}.

\subsection{Regular Expressions}\label{sec:regexs}
The \term{Chomsky hierarchy} is a classification of text production rules
(called \termpl{formal grammar}), which was proposed \cite{chomsky56} in
\citeyear{chomsky56} by the American linguist \person{Noam Chomsky} in his
endeavor to discover a good formal model for the description of natural
languages. The class of \termpl{regular grammar}, which is the least powerful
of the proposed classes, has properties that make it possible to determine the
grammaticality of a text in constant time and space. \Termpl{regular
expression} are a more intuitive reformulation of regular grammars that a writer
can use to find and replace character strings within text.

Since regular expressions are just a formal model, a software implementation
needs to settle on a concrete syntax. One of the earliest standard syntaxes are
\acronym{BRE} and \acronym{ERE} \cite[part~1,~ch.\,9]{iso93:posix2}, which are
supported by most utilities using regular expressions on \Unices. Both syntaxes
are described in Table \ref{tab:regexs}. More extensive syntaxes include the
\acroshort{GNU} extensions of \acronym{BRE} and \acronym{ERE}, the regex syntag
of the \inx{Perl} programming language, and their dialects and derivatives. For
most of these extended syntaxes, the term \term*{regular} is a misnomer, as the
expressions can be used to build grammars that, according to the Chomsky
hierarchy, aren't regular.\footnote{
  A curious reader with a background in formal language theory should direct
  their attention to Perl self-referencing groups and look-aheads, which make it
  a simple matter to create expressions that match non-context-free formal
  languages.
% It is easy to show that the Perl 5 regex of \code{(?=(a(?-1)?b)c)
% a+(b(?-1)?c)} generates the context-sensitive language of $a^nb^nc^n$ for
% arbitrary characters $a,b,c$ with superscripts denoting repetition.
} To disambiguate the term, these expressions are often called
\termpl*[regexes]{regex}\index{regex|see{regular expression}}.

\begin{table}
  \input examples/01/regexs
  \label{tab:regexs}
\end{table}

%%% POSIX.2-1997: Regular Expressions
%%%   <http://pubs.opengroup.org/onlinepubs/007908799/xbd/re.html>

Many regex syntaxes and the software that implements them were designed for the
processing of \acronym{ASCII} text and may behave in surprising ways, when
confronted with \acronym{UCS} characters. The software may assume that each
character is exactly one byte long and fail to recognize any characters that
occupy several bytes as a single character, or it may assume that all
\acronym{UCS} characters fall within \acronym{BMP} and exhibit the same problem
with characters outside \acronym{BMP}. More subtle, but no less precarious, can
be the lack of support for Unicode case conversion and normalization algorithms,
which makes it difficult to perform robust case-insensitive matching and the
matching of characters that can be encoded in several different ways. The lack
of awareness of the invisible characters that can appear in \acronym{UCS}
text---such as the \ucs[200B]{zero width space}, \ucs[200C]{zero width
non-joiner}, \ucs[200D]{zero width joiner}, and \ucs[FEFF]{zero width no-break
space}---, is also problematic and can lead to false negative matches.
Conversely, modern regex syntaxes that at least partially implement the Unicode
standard for Regular Expressions \cite{unicode13}---such as those of Perl 5.2 or
Java 7---are actively aware of \acronym{UCS} and provide features that enable
the matching of characters based on their general category, numeric value,
directionality, and other properties defined by Unicode, as shown in Table
\ref{tab:unicode-regexs}.

\begin{table}[!tb]
  \input examples/01/unicode-regexs
  \caption{An overview of the elements of the Unicode regex syntax as
    implemented by Perl 5.2 and Java 7. The list of Unicode character properties
    is only demonstrative.}
  \label{tab:unicode-regexs}
\end{table}

The most elementary text processing \acroshort{CLI} tool is \cliutil{grep}%
\footnote{
  The authoritative resource on \cliutil{grep}, \cliutil{sed}, and \cliutil{awk}
  is \citework{dougherty97}, which explains each tool as well as the
  \acroshort{BRE} and \acroshort{ERE} syntaxes in full detail.
}, which makes it possible to search text files for fixed strings and regexes in
default of an advanced text editor. Unless configured otherwise, the tool will
present lines that contain one or more matches to the user. A more advanced
text-processing \acroshort{CLI} tool is \cliutil{sed}, which features a simple
programming language that can be used to arbitrarily transform text files.
\Cliutil{awk} is a \acroshort{CLI} tool that also features a text-processing
programming language, albeit a more advanced one than that of \cliutil{sed}.
Originally developed for the Research Unix during 1973--1977, \cliutil{grep},
\cliutil{sed}, and \cliutil{awk} are available in various flavors for most
operating systems.

%%% The true power of regular expressions
%%%   <https://nikic.github.io/2012/06/15/The-true-power-of-regular-expressions.html>
%%%
%%% Regular Expression Matching Can Be Simple And Fast
%%%   <https://swtch.com/~rsc/regexp/regexp1.html>
%%%
%%% JavaScript has a Unicode problem
%%%   <https://mathiasbynens.be/notes/javascript-unicode>

\section{Version Control}\label{sec:vcs}
When writing a text document, it is ofter useful to have a backup of the
previous versions of files, so that undesirable changes can be reverted whenever
necessary. If more than one person contributes to the document, the ability to
track the authorship of these changes also becomes an asset. At its most
rudimentary, \acronym{VCS} records changes along with a short description and
information about their author. These changes can then be browsed, and reverted.
With a single contributor, \acronym{VCS} are a convenient alternative to manual
version archival. With several contributors, \acronym{VCS} becomes an essential
tool.

\Acronym{VCS} can be dichotomized based on its architecture, which is either
\term*{centralized}\acroindex[!centralized]{VCS} or \term*{decentralized}
\acroindex[!decentralized]{VCS}. Centralized \acronym{VCS} stores
all versions in a repository located on a remote server. Users send new versions
to the server and retrieve existing versions using a client software. The client
software is \term*{thin} in the sense that it does not store more than one
version locally and its operation is fully dependent on the availability of the
server. An example of centralized \acronym{VCS} is \acronym{SVN}\footnote{
  The authoritative resource on \acronym{SVN} is \citework{sussman02},
  affectionately known as \work{the Subversion book}.
}.

By comparison, there is no designated server in decentralized \acronym{VCS} and
the users can send and download new versions directly from one another. The
client software is \term*{thick} in the sense that all users have a local
repository with every existing version, which they can browse and manipulate
at any time. The disadvantages of decentralization include the more complex
workflow, greater storage size requirements and the increased opportunity for
the users not to share their local changes frequently enough, leading to an
increased chance of collisions. Examples of decentralized \acronym{VCS} include
\inx{Git}, \inx{Mercurial}, or \inx{Bazaar}.

\begin{figure}
  \input examples/01/svn
\end{figure}

\begin{figure}
%  \ffootnote{Although it is typical to use a central repository, the
%    decentralized architecture of Git makes it possible for clients to exchange
%    the contents of their repositories directly.  Multiple layers of
%    repositories that automatically exchange the latest updates can also be
%    created for backup and other purposes.}
  \input examples/01/git
\end{figure}

%\begin{figure}
%  \input examples/01/git-svn
%\end{figure}

Although \acronym{VCS} can be used to keep track of any kind of files, they are
especially geared towards text files, which they can easily display along with
changes. However, most interactive \acropl{DPS} do not produce text files, which
can make version control challenging. As a solution, some \acropl{DPS} include
an internal version control functionality that can record changes directly into
output files. Other \acropl{DPS} provide an interface for external \acronym{VCS}
to display changes between two versions of output documents produced by the
\acropl{DPS}\footnote{
  An example would be the graphical \acroshort{SVN} client \inx{Tortoise
  \acroshort{SVN}} that is able to display the changes between two versions
  of Microsoft Word documents using the interface provided by Microsoft Office.
}. A class of its own are web services that enable real-time interactive
collaboration---such as \inx{Word Online} or \inx{Google Documents}.

\begin{figure}
  \includegraphics[width=\textwidth]{examples/01/word.png}\nextimage
  \includegraphics[width=\textwidth]{examples/01/openoffice.png}
  \caption{The built-in \acronym{VCS} of \inx{Microsoft Word} (above) and
    \inx{Apache OpenOffice} (below)}
\end{figure}

\begin{figure}
  \includegraphics[width=\textwidth]{examples/01/tortoise-svn.png}
  \caption{\inx{Tortoise \acroshort{SVN}} is a graphical frontend for
    \acronym{SVN} with the ability to display the difference between two versions
    of a \inx{Microsoft Word} document even though it is not a text file.}
\end{figure}

%%% Git porcelains <http://stackoverflow.com/a/6978402>
%%% Git <-> SVN
%%%   <https://git-scm.com/book/en/v1/Git-and-Other-Systems-Git-and-Subversion>
%%%   <http://www.janosgyerik.com/practical-tips-for-using-git-with-large-subversion-repositories/>
%%%   <http://stackoverflow.com/a/772881/657401>
