{\input examples/01/regexs-definitions
\tablebegin
  \tblhead{\normalsize\acroshort{BRE} regex} &
    \tblhead{\normalsize Description} &
    \tblhead{\normalsize Matches} \\ \hline
    \regex{we\{1,2\}p} & The repetition expression in the form of
      $c$\regex{\{}$m$\regex{,}$n$\regex{\}} matches the character $c$ repeated
      \mbox{$k$ $\in\langle $$m$$;$\,$n$$\rangle$} times.  Other forms include
      $c$\regex{\{}$m$\regex{,}\regex{\}} for $k$ $\in\langle $$m$$; \infty)$
      and $c$\regex{\{}$m$\regex{\}} for $k$ = $m$. &
      \match{weep}s, \match{wep}t\\
  \regex{e*ne} & Star (\regex{*}) is a \regexTerm{repetition
    operator} equivalent to the interval expression of \regex{\{0,\}}. &
    \match{ne}ver, \match{ene}my, Kl\match{eene} \\
  \regex{\(}\meta{regex}\regex{\)} &
    A \regexTerm{subexpression} is a parenthesized regex. Any interval
    expression or repetition operator used immediately after a
    subexpression applies to the entire parenthesized regex. &
    \meta{regex} \\
  \regex{^ar} & 
    At the beginning of a regex or a subexpression, a caret (\regex{^})
    matches the beginning of a string. &
    \match{ar}gument, \match{ar}row keys, \match{ar}a ararauna \\
  \regex{ore$} &
    At the end of a regex or a subexpression, the dollar sign (\regex{$})
    matches the end of a string. &
    iron \match{ore}, dumbled\match{ore} \\
  \regex{be.} &
    A period (\regex{.}) matches any single character. &
    or not to \match{be?} \\
  \regex{be[ea]} &
    A \regexTerm{matching list expression} is enclosed in square brackets
    (\regex{[ ]}) and contains a list of characters that the bracket
    expression matches. It may contain other entities omitted here for brevity. &
    \match{bee}hive, grizzly
    \match{bea}r, glass \match{bea}ds \\
  \regex{be[^ea]} &
    A \regexTerm{non-matching list expression} contains a caret (\regex{^}) as
    its first character and matches any character that the corresponding
    matching list expression would not match. & o\match{bea}h, \match{ben}d,
    li\match{bel}a \\
  \regex{\^\*\.\\\$} & Backslash (\regex{\ }) is an
    \regexTerm{escape character} that either suppresses or activates the
    special meaning of the following character. &
    \match{\textasciicircum*.\textbackslash\$} \\
  \regex{\(..\).*\1} & A \regexTerm{backreference} in the form of an
    escaped number $n\in\langle1;9\rangle$ (\regex{\1}, \regex{\2}, \ldots,
    \regex{\9}) matches anything the $n$th subexpression matched.
    & \match{ara ara}rauna, \match{darda}nelles, \match{nationa}lity \\
\end{tabularx}}
\separatorcaption{An informal description of the \acroshort{BRE} syntax
  (above) and the differences in the \acroshort{ERE} syntax (below)}
{\input examples/01/regexs-definitions
\tablebegin
  \tblhead{\normalsize\acroshort{ERE} regex} &
    \tblhead{\normalsize Description} &
    \tblhead{\normalsize Matches} \\ \hline
    \regex{we{1,2}p} & Unlike in \acropl{BRE}, braces aren't escaped. &
      \match{weep}s, \match{wep}t\\
  \regex{pe+rl?} & The plus sign (\regex{+}) and the question mark (\regex{?})
    are repetition operators equivalent to the interval expressions of
    \regex{\{1,\}} and \regex{\{0,1\}}. &
    \match{per}sona, \match{peer}, s\match{pee}ch, \match{perl} \\
  \regex{(}\meta{regex}\regex{)} &
    Unlike in \acropl{BRE}, parentheses aren't escaped. & \meta{regex} \\
  \regex{(on|t).} & Vertical line (\regex{|}) is an
    \regexTerm{alternation operator} that separates multiple regexes. The
    whole regex matches any of the alternative regexes. &
    \match{one}, \match{tw}o, \match{tr}ophy, \match{tr}uth \\
  \regex{(..).*\1} & \acropl{ERE} do not support backreferences. &
    \meta{undefined} \\
\end{tabularx}}
