{\input examples/01/vcs-definitions
\begin{tikzpicture}
  \svnnodes
  \path (E) edge[loop above, in=70, out=110, min distance=.05\textwidth, ->]
            node[above=\serverposabove] {\command{svnadmin create}} (E)
        (A) edge[bend left=\extrabend] (E) (E) edge[bend left=\extrabend] (A)
        (B) edge[bend left=\bend] (A) (A) edge[bend left=\bend, ->] (B)
        (C) edge[bend left=\bend] (A) (A) edge[bend left=\bend, ->]
            node[rotate=45,above=\clientposabove]
              {\command{svn checkout}} (C)
        (D) edge[bend left=\bend] (A) (A) edge[bend left=\bend, ->] (D);
\end{tikzpicture}\hfill\begin{tikzpicture}
  \svnnodes
  \path (B) edge[\svncheckout,bend left=\bend] (A) (A) edge[\svncheckout,bend left=\bend, ->] (B)
        (A) edge[\svncheckout,bend left=\extrabend] (E)
        (E) edge[\svncheckout,bend left=\extrabend] (A)
        (C) edge[\svncheckout,bend left=\bend] (A) (A) edge[\svncheckout,bend left=\bend, ->]
            node[\svncheckout,rotate=45,above=\clientposabove]
              {\command{svn update}} (C)
        (D) edge[\svncheckout,bend left=\bend]
            node[\svncommit,rotate=-45,above=\clientposabove]
              {\command{svn commit}} (A) (A) edge[\svncheckout,bend left=\bend, ->] (D)
        (B) edge[\svncommit] (A) (C) edge[\svncommit] (A) (D) edge[\svncommit] (A) (A)
            edge[\svncommit, ->] (E);
\end{tikzpicture}
\caption{The basic \acroshort{SVN} workflow. After a remote repository has been
  created, users download the latest version of the document and then
  \textcolor{\svncheckout}{keep downloading the latest changes by other users} and
  \textcolor{\svncommit}{uploading changes of their own}.}}
