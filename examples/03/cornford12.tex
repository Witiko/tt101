\documentclass[11pt]{article}
\usepackage{fontspec, leading, newunicodechar}
\usepackage[Latin, Greek]{ucharclasses}
\setTransitionsForLatin{%
  \fontspec{AlegreyaSans-Regular.ttf}[Ligatures=TeX]}{}
\setTransitionsForGreek{%
  \fontspec{GFSNeohellenic.otf}[Scale=1.2, WordSpace=0.5, Ligatures=TeX]}{}
\newunicodechar{·}{\raisebox{.8ex}{.}}
\frenchspacing
\leading{14pt}

\begin{document}
  The second function of Soul -- knowing -- was not at
  first distinguished from motion. Aristotle says, φαμὲν
  γὰρ τὴν ψυχὴν λυπεῖσθαι χαίρειν, θαρρεῖν φοβεῖσθαι, ἔτι
  δὲ ὸργίζεσθαί τε καὶ αἰσθάνεσθαι καὶ διανοεῖσθαι· ταῦτα
  δὲ πάντα κινήσεις εἶναι δοκοῦσιν. ὅθεν οἰηθείη τις ἂν
  αὐτὴν κινεῖσθαι.
  ``The soul is said to feel pain and joy, confidence and
  fear, and again to be angry, to perceive, and to think;
  and all these states are held to be movements, which
  might lead one to suppose that soul itself is moved.''
\end{document}
